\chapter{Σχετική Δουλειά}
\label{ch:relativeWork}
\section{Pattern Wizard}
\label{sec:patternWizard}
Αυτό το εργαλείο μετατρέπει τον βασικό κώδικα σε μία μορφή κώδικα που χρησιμοποιεί ένα σχεδιαστικό πρότυπο, 
ένας προγραμματιστής μπορεί να αναλύσει τον τρόπο που λειτουργεί ένα μοτίβο, χωρίς να χρειάζεται να υλοποιήσει τον κώδικα.
Το εργαλείο αυτό παρέχει τα μοτίβα Adapter, Abstract factory και Observer για την γλώσσα προγραμματισμού java.
\section{Patternbox}
Το εργαλείο αυτό δημιουργεί ένα ειδικό αρχείο το οποίο αναπαριστά το μοτίβο. Μέσα από το αρχείο μπορεί κάποιος να 
επιλέξει κάποιο στοιχείο του μοτίβου και να επιλέξει την τοποθεσία του αντικειμένου. Ένα αντικείμενο μπορεί να είναι είτε
μία κλάση είτε μία διεπαφή. Αφού ο χρήστης ολοκληρώσει την διαδικασία, το εργαλείο παράγει ένα πρότυπο κώδικα.
\label{sec:Patternbox}
\section{DESIGN PATTERN AUTOMATION TOOLKIT}
\label{sec:dpa}
Το εργαλείο αυτό  εμφανίζει τη χρήση προτύπων σχεδίασης σε πολλές γλώσσες. Διαθέτει επίσης έναν τρόπο
δημιουργίας νέων προτύπων σχεδίασης (εκτός από τα πρότυπα των GoF) από την πλευρά του χρήστη 
χρησιμοποιώντας μια γλώσσα περιγραφής προτύπων. Ωστόσο, αυτό το εργαλείο δεν τροποποιεί τον υπάρχοντα κώδικα ώστε 
να ταιριάζει με το πρότυπο, απλώς τον παράγει.
\section{ALPHAWORKS DESIGN PATTERN TOOLKIT}
\label{sec:ALPHAWORKS}
Το εργαλείο αυτό είναι μία επέκταση του Eclipse και μοιάζει με το εργαλείο \ref{sec:Patternbox}, επίσης και αυτό το εργαλείο δεν τροποποιεί
τον κώδικα.
\footnote{\cite{patternBox}}