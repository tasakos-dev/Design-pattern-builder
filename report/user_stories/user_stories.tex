\documentclass[../diploma_thesis.tex]{subfiles}


\begin{document}
\begin{table}[H]
	\hspace*{-4.5cm}
	\begin{adjustbox}{width=1.7\textwidth}
    \centering
    \begin{tabular}{|l|l|l|l|}
    \hline
        USER STORY ID & AS A [type of user] & I NEED TO [do some task] & SO THAT I CAN [get some result] \\ \hline
        US1 & Προγραμματιστής & Να μπορώ να επιλέξω κατηγορία μοτίβων. & Έτσι ώστε να εισάγω αυτόματα στον κώδικά μου το σκελετό ενός μοτίβου της κατηγορίας αυτής. \\ \hline
        US2 & Προγραμματιστής & Να μπορώ να επιλέγω ένα μοτίβο μιας κατηγορίας. & Έτσι ώστε να  εισάγω αυτόματα στον κώδικά μου το σκελετό του μοτίβου αυτού. \\ \hline
        US3 & Προγραμματιστής & Να μπορώ να καθορίσω τα ονόματα των κλάσεων που θα δημιουργηθούν αυτόματα. & Έτσι ώστε να προσαρμοσω τις κλάσεις αυτές στον κώδικά μου και στις ανάγκες του μοτίβου. \\ \hline
        US4 & Προγραμματιστής & Να μπορώ να επιλέξω ήδη υπάρχουσες κλάσεις που υπάρχουν στον κώδικα μου.  & Έτσι ώστε να παίξουν κάποιο ρόλο στο μοτίβο. \\ \hline
        US5 & Προγραμματιστής & Να μπορώ να καθορίσω πεδία που θα προστεθούν στις νέες κλάσεις.  & Έτσι ώστε να προσαρμοσω τις κλάσεις αυτές στον κώδικά μου και στις ανάγκες του μοτίβου. \\ \hline
        US6 & Προγραμματιστής & Να μπορώ να καθορίσω μεθόδους που θα προστεθούν στις νέες κλάσεις. & Έτσι ώστε να προσαρμοσω τις μεθόδους αυτές στον κώδικά μου και στις ανάγκες του μοτίβου. \\ \hline
        US7 & Προγραμματιστής & Να μπορώ να καθορίσω πεδία που υπάρχουν ήδη στις υπάρχουσες κλάσεις. & Έτσι ώστε να προσαρμοσω τις κλάσεις αυτές στον κώδικά μου και στις ανάγκες του μοτίβου. \\ \hline
        US8 & Προγραμματιστής & Να μπορώ να καθορίσω μεθόδους που υπάρχουν ήδη στις υπάρχουσες κλάσεις. & Έτσι ώστε να προσαρμόσω τις κλάσεις αυτές στον κώδικά μου και στις ανάγκες του μοτίβου. \\ \hline
        US9 & Προγραμματιστής & Να μπορώ να δημιουργήσω αυτόματα τον κώδικα του μοτίβου με βάση τις όποιες παραμετροποιήσεις έχουν γίνει. & Έτσι ώστε να εισάγω αυτόματα στον κώδικά μου το σκελετό του μοτίβου. \\ \hline
        US10 & Προγραμματιστής & Να μπορώ να καθορίσω τα ονόματα των διεπαφών που θα δημιουργηθούν αυτόματα. & Έτσι ώστε να προσαρμοσω τις διεπαφές αυτές στον κώδικά μου και στις ανάγκες του μοτίβου. \\ \hline
        US11 & Προγραμματιστής & Να μπορώ να επιλέξω ήδη υπάρχουσες διεπαφές που υπάρχουν στον κώδικα μου.  & Έτσι ώστε να παίξουν κάποιο ρόλο στο μοτίβο. \\ \hline
        US12 & Προγραμματιστής & Να μπορώ να καθορίσω μεθόδους που θα προστεθούν στις νέες διεπαφές. & Έτσι ώστε να προσαρμοσω τις διεπαφές αυτές στον κώδικά μου και στις ανάγκες του μοτίβου. \\ \hline
        US13 & Προγραμματιστής & Να μπορώ να καθορίσω μεθόδους που υπάρχουν ήδη στις υπάρχουσες διεπαφές. & Έτσι ώστε να προσαρμοσω τις διεπαφές αυτές στον κώδικά μου και στις ανάγκες του μοτίβου. \\ \hline
        US14 & Προγραμματιστής & Να μπορώ να ακυρώσω την διαδικασία. & Έτσι ώστε να επιστρέψω σε αυτό που έκανα χωρίς να αλλάξω τον κώδικα μου. \\ \hline
    \end{tabular}
    \end{adjustbox}
\end{table}
\end{document}