\chapter{Σχετική Δουλειά}
\label{ch:relativeWork}
\section{Σχετικά εργαλεία}
\label{sec:relativeTools}
% \subsection{Pattern Wizard}
% \label{subsec:patternWizard}
\textbf{Pattern Wizard.} Το εργαλείο αυτό \cite{PatternBox}, προσφέρει ως λειτουργίες, την επιλογή κάποιου μοτίβου 
από τα Adapter, Abstract Factory και Observer, καθώς και την επιλογή γλώσσας προγραμματισμού, μέχρι στιγμής την Java. Στην συνέχεια 
ο χρήστης έχει την δυνατότητα να αντιστοιχίσει υπάρχων κώδικά στο μοτίβο που έχει επιλέξει και το εργαλείο μετατρέπει τον κώδικα 
σε κώδικα που είναι συμβατός με το μοτίβο. Τέλος, ο προγραμματιστής έχει την δυνατότητα να επιλέξει κάποιο κενό πηγαίο αρχείο και 
το εργαλείο εξάγει τον σκελετό του μοτίβου.
\par
% \subsection{Patternbox}
% \label{subsec:Patternbox}
\textbf{Patternbox.} Το εργαλείο αυτό \cite{PatternBox} υποστηρίζει 16 μόνο 
μοτίβα από αυτά που έχουν προτείνει οι GoF \cite{GoF}, δημιουργεί ένα ειδικό 
αρχείο το οποίο αναπαριστά το μοτίβο. Μέσα από το αρχείο μπορεί ο προγραμματιστής 
να επιλέξει κάποιο στοιχείο του μοτίβου και να επιλέξει την τοποθεσία όπου 
θα εξαχθεί είτε η κλάση είτε η διεπαφή που έχει επιλέξει ο χρήστης. 
Αφού ο χρήστης ολοκληρώσει την διαδικασία, 
το εργαλείο παράγει απλώς τον σκελετό του μοτίβου, χωρίς να τροποποιεί τον κώδικα.
\par
% \subsection{Design Pattern Automation Toolkit}
% \label{subsec:dpa}
\textbf{Design Pattern Automation Toolkit.} Το εργαλείο αυτό \cite{PatternBox} 
υποστηρίζει και τα 23 μοτίβα των GoF \cite{GoF}, υποστηρίζει λειτουργίες 
όπως επιλογή μοτίβου και γλώσσας προγραμματισμού. 
Επίσης κάποιος προγραμματιστής μπορεί να εισάγει νέα μοτίβα που μπορεί 
να υλοποιήσει το εργαλείο.Τέλος δεν τροποποιεί τον υπάρχων κώδικά 
 παρά μόνο εξάγει τον σκελετό του μοτίβου.
\par
% \subsection{Alphaworks Design Pattern Toolkit}
% \label{subsec:ALPHAWORKS}
\textbf{Alphaworks Design Pattern Toolkit.} Αυτό το εργαλείο \cite{PatternBox} 
είναι αντίστοιχο του Patternbox, με την διαφορά ότι το Alphaworks μετασχηματίζει 
τον κώδικα του προγραμματιστή σε ένα ενδιάμεσο αρχείο τύπου xml, 
αφού ο προγραμματιστής έχει εισάγει στον κώδικά του κατάλληλα tags. 
Στη συνέχεια, ο προγραμματιστής κάνει τις αλλαγές που επιθυμεί στο αρχείο xml 
και το εργαλείο μετασχηματίζει πίσω σε κώδικα συμβατό με το μοτίβο.
\subsection{Σύγκριση}
\label{subsec:compare}
Το Design Pattern Builder, το εργαλείο που αποτελεί αντικείμενο 
της διπλωματικής αυτής, παρουσιάζει πλεονεκτήματα έναντι των παραπάνω εργαλείων.
Ένα από τα πλεονεκτήματα του εργαλείου είναι ότι ο προγραμματιστής μπορεί να τροποποιήσει οποιοδήποτε μοτίβο, 
ώστε το μοτίβο να ταιριάξει με τις ανάγκες του έργου του προγραμματιστή. 
Επίσης, εξάγει annotations -τα οποία αποδίδουν τον ρόλο που έχει η κλάση ή η διεπαφή-  
στον σκελετό του μοτίβου ώστε να διευκολύνει τον προγραμματιστή 
στην εφαρμογή του μοτίβου. 
Τέλος, το εργαλείο που προτείνουμε προσφέρει και τα 23 μοτίβα των GoF \cite{GoF}.
\par
Από την άλλη, το εργαλείο που προτείνουμε υστερεί ως προς την δυνατότητα 
αντιστοίχισης ενός στοιχείου, δηλαδή μίας κλάσης ή μίας διεπαφής, του μοτίβου με κάποιο στοιχείο 
του έργου του προγραμματιστή. 
% Τέλος ένα ακόμα μειονέκτημα είναι η δυνατότητα 
% εξαγωγής κώδικα και σε άλλες γλώσσες προγραμματισμού εκτός της java.
\section{Υπόβαθρο}
\label{sec:background}
Τα σχεδιαστικά μοτίβα GoF (Gang of Four) είναι ένα σύνολο σχεδιαστικών 
προτύπων που περιγράφονται στο βιβλίο \cite{GoF}. 
Αυτά τα πρότυπα προέκυψαν από την εμπειρία και την εμπειρογνωμοσύνη των 
τεσσάρων συγγραφέων, και αποτελούν γενικές λύσεις για συχνά προβλήματα 
στον σχεδιασμό λογισμικού. Οι GoF πρότειναν 23 μοτίβα στο πλαίσιο της γλώσσας 
προγραμματισμού C++, αλλά μπορούν να εφαρμοστούν και σε άλλες γλώσσες 
αντικειμενοστραφούς προγραμματισμού όπως η Java. \par
Για να περιγράψουμε ένα σχεδιαστικό μοτίβο, δεν αρκεί μία απλή γραφική 
αναπαράσταση, καθώς απλώς αποτυπώνει τις σχέσεις μεταξύ κλάσεων και αντικειμένων. 
Για να μπορέσουμε να επαναχρησιμοποιήσουμε την σχεδίαση πρέπει να μπορούμε 
να καταγράψουμε τις αποφάσεις, τους συμβιβασμούς που χρειάζεται 
να κάνουμε για να εφαρμόσουμε την σχεδίαση αυτή, καθώς και τις εναλλακτικές λύσεις. 
Χρειάζονται επίσης, παραδείγματα, καθώς αναδεικνύουν την εφαρμογή της σχεδίασης 
αυτής. Για την περιγραφή των μοτίβων χρησιμοποιείται μία συνεπής μορφή. 
Κάθε μοτίβο διαιρείται σε ενότητες σύμφωνα με το πρότυπο που θα αναλυθεί παρακάτω. 
Το πρότυπο προσδίδει μια ομοιόμορφη δομή στις πληροφορίες, 
διευκολύνοντας την εκμάθηση, τη σύγκριση, καθώς και την χρήση των 
σχεδιαστικών μοτίβων.
\begin{itemize}
    \item \textbf{Όνομα και κατηγορία μοτίβου:} Το όνομα αποδίδει συνοπτικά την ουσία του μοτίβου.
    \item \textbf{Πρόθεση:} Τι κάνει το μοτίβο ποία είναι η λογική και η πρόθεση του. Ποιο σχεδιαστικό πρόβλημα αντιμετωπίζει το μοτίβο.
    \item \textbf{Γνωστό και ως:} Άλλα γνωστά ονόματα για το μοτίβο, εάν υπάρχουν.
    \item \textbf{Κίνητρα:} Ένα σενάριο που απεικονίζει ένα σχεδιαστικό πρόβλημα και τον τρόπο με τον οποίο οι
    δομές κλάσεων και αντικειμένων στο μοτίβο λύνουν το πρόβλημα. Το σενάριο βοηθάει να κατανοηθεί
    αφηρημένη περιγραφή του προτύπου που ακολουθεί.
    \item \textbf{Εφαρμογές:} Σε ποιες περιπτώσεις μπορεί να εφαρμοστεί το πρότυπο σχεδίασης. Ποια είναι
    τα παραδείγματα κακής σχεδίασης που μπορεί να αντιμετωπίσει το πρότυπο.
    \item \textbf{Δομή:} Ένα διάγραμμα OMT \cite{citeulike:348271}, των κλάσεων του 
    προτύπου και διαγράμματα αλληλεπίδρασης \cite{jacobson92usecase} για την απεικόνιση ακολουθίες αιτημάτων και συνεργασίες μεταξύ αντικειμένων.
    \item \textbf{Συμμετέχοντες:} Οι κλάσεις και/ή τα αντικείμενα που συμμετέχουν στο πρότυπο σχεδίασης και οι
    αρμοδιότητές τους.
    \item \textbf{Συνεργασίες:} Πώς οι συμμετέχοντες συνεργάζονται για την εκτέλεση των αρμοδιοτήτων τους.
    \item \textbf{Συνέπειες:} Πώς η σχεδίαση υποστηρίζει τους στόχους της. Ποια είναι τα αντισταθμιστικά οφέλη και τα αποτελέσματα της 
    χρήσης του προτύπου.
    \item \textbf{Υλοποίηση:} Ποιες παγίδες, τεχνικές θα πρέπει να γνωρίζει κάποιος κατά την εφαρμογή
    του προτύπου. Υπάρχουν θέματα που αφορούν συγκεκριμένες γλώσσες.
    \item \textbf{Παραδείγματα:} Τμήματα κώδικα που απεικονίζουν τον τρόπο με τον οποίο μπορεί να υλοποιηθεί
    το πρότυπο σε C++ ή Smalltalk.
    \item \textbf{Γνωστές χρήσεις:} Παραδείγματα του μοτίβου που συναντώνται σε πραγματικά συστήματα. Περιλαμβάνονται τουλάχιστον 
    δύο παραδείγματα από διαφορετικούς τομείς.
    \item \textbf{Σχετικά μοτίβα:} Ποια πρότυπα σχεδίασης σχετίζονται με αυτό. Ποιες είναι οι σημαντικές
    διαφορές. Με ποια άλλα πρότυπα θα πρέπει να χρησιμοποιείται αυτό το πρότυπο.
\end{itemize}
\par
Τα μοτίβα που πρότειναν οι GoF, χωρίζονται σε τρεις κατηγορίες, 
\begin{itemize}
    \item \textbf{Δημιουργικά πρότυπα:} Για την δημιουργία αντικειμένων.
    \item \textbf{Δομικά πρότυπα:} Για την δημιουργία σχέσεων μεταξύ αντικειμένων.
    \item \textbf{Πρότυπα συμπεριφοράς:} Για τον καθορισμό του τρόπου αλληλεπίδρασης μεταξύ αντικειμένων.
\end{itemize}
\par
Τέλος, για να επιλέξει κάποιος προγραμματιστής το μοτίβο ανάμεσα σε παραπάνω από 
20 μοτίβα που ταιριάζει στο πρόβλημα του, προτείνεται: 
\begin{itemize}
    \item Να εξετάσει πως ένα μοτίβο επιλύει ένα σχεδιαστικό πρόβλημα.
    \item Να διαβάσει τις ενότητες πρόθεσης.
    \item Να μελετήσει το πως τα σχεδιαστικά πρότυπα αλληλεπιδρούν.
    \item Να μελετήσει μοτίβα με παρόμοιο σκοπό.
    \item Να εξετάσει τις αιτίες επανασχεδιασμού του έργου του.
    \item Να σκεφτεί τι πρέπει να αλλάξει στο έργο του.
\end{itemize}
