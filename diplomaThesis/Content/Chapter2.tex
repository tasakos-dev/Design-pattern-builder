\chapter{Σχετική Δουλειά}
\label{ch:relativeWork}
\section{Σχετικά εργαλεία}
\label{sec:relativeTools}
\subsection{Pattern Wizard}
\label{subsec:patternWizard}
Το εργαλείο\textsubscript{\cite{PatternBox}} αυτό, προσφέρει ως λειτουργίες, την επιλογή κάποιου μοτίβου 
από τα Adapter, Abstract Factory και Observer, καθώς και την επιλογή γλώσσας προγραμματισμού, μέχρι στιγμής την Java. Στην συνέχεια 
ο χρήστης έχει την δυνατότητα να αντιστοιχίσει υπάρχων κώδικά στο μοτίβο που έχει επιλέξει και το εργαλείο μετατρέπει τον κώδικα 
σε κώδικα που είναι συμβατός με το μοτίβο. Τέλος, ο προγραμματιστής έχει την δυνατότητα να επιλέξει κάποιο κενό πηγαίο αρχείο και 
το εργαλείο εξάγει τον σκελετό του μοτίβου.
\subsection{Patternbox}
\label{subsec:Patternbox}
Το εργαλείο\textsubscript{\cite{PatternBox}} αυτό υποστηρίζει 16 μόνο μοτίβα από αυτά που έχουν προτείνει οι GoF\textsubscript{\cite{GoF}} δημιουργεί ένα ειδικό 
αρχείο το οποίο αναπαριστά το μοτίβο. Μέσα από το αρχείο μπορεί ο προγραμματιστής να επιλέξει κάποιο στοιχείο του μοτίβου και 
να επιλέξει την τοποθεσία οου θα εξαχθεί είτε η κλάση είτε η διεπαφή που έχει επιλέξει ο χρήστης. 
Αφού ο χρήστης ολοκληρώσει την διαδικασία, το εργαλείο παράγει απλώς τον σκελετό του μοτίβου, χωρίς να τροποποιεί τον κώδικα.
\subsection{Design Pattern Automation Toolkit}
\label{subsec:dpa}
Το εργαλείο\textsubscript{\cite{PatternBox}} αυτό υποστηρίζει και τα 23 μοτίβα των GoF\textsubscript{\cite{GoF}}, υποστηρίζει λειτουργίες 
όπως επιλογή μοτίβου και γλώσσας προγραμματισμού. Επίσης κάποιος προγραμματιστής μπορεί να εισάγει νέα μοτίβα που μπορεί να υλοποιεί το εργαλείο.
Τέλος δεν τροποποιεί τον υπάρχων κώδικά και παρά μόνο εξάγει τον σκελετό του μοτίβου.
\subsection{Alphaworks Design Pattern Toolkit}
\label{subsec:ALPHAWORKS}
Αυτό το εργαλείο\textsubscript{\cite{PatternBox}} είναι αντίστοιχο του Patternbox\textsubscript{\ref{sec:Patternbox}}, με την διαφορά ότι το Alphaworks μετασχηματίζει τον κώδικα
του προγραμματιστή σε ένα ενδιάμεσο αρχείο τύπου xml, αφού ο προγραμματιστής έχει εισάγει στον κώδικά του κατάλληλα tags.
Στη συνέχεια ο προγραμματιστής κάνει τις αλλαγές που επιθυμεί στο αρχείο xml και το εργαλείο μετασχηματίζει πίσω σε κώδικα συμβατό με το μοτίβο.
\subsection{Σύγκριση}
\label{subsec:compare}
Το Design Pattern Builder, το οποίο είναι το εργαλείο που αποτελεί αντικείμενο της διπλωματικής αυτής, παρουσιάζει πλεονεκτήματα έναντι των παραπάνω εργαλείων,
καθώς ο προγραμματιστής μπορεί να τροποποιήσει οποιοδήποτε μοτίβο, ώστε το μοτίβο να ταιριάξει με τις ανάγκες του έργου του προγραμματιστή. 
Επίσης εξάγει annotations -τα οποία αποδίδουν τον ρόλο που έχει η κλάση ή διεπαφή-  στον σκελετό του μοτίβου ώστε 
να διευκολύνει τον προγραμματιστή στην εφαρμογή του μοτίβου. Τέλος το εργαλείο που προτείνουμε προσφέρει και τα 23 μοτίβα 
των GoF\textsubscript{\cite{GoF}}.
\linebreak
\linebreak
Από την άλλη, το εργαλείο που προτείνουμε υστερεί ως προς την δυνατότητα αντιστοίχισης 
ενός στοιχείου του μοτίβου με κάποιο στοιχείο του έργου του προγραμματιστή. Τέλος ένα ακόμα μειονέκτημα 
είναι η δυνατότητα εξαγωγής κώδικα και σε άλλες γλώσσες προγραμματισμού εκτός της java.
\section{Υπόβαθρο}
\label{sec:background}
Τα σχεδιαστικά μοτίβα GoF (Gang of Four), είναι ένα σύνολο σχεδιαστικών προτύπων που περιγράφονται στο βιβλίο \cite{GoF}.
Αυτά τα πρότυπα προέκυψαν από την εμπειρία και την εμπειρογνωμοσύνη των τεσσάρων συγγραφέων, οι GoF, και αποτελούν γενικές λύσεις 
για συχνά προβλήματα στον σχεδιασμό λογισμικού. Οι GoF πρότειναν 23 μοτίβα στο πλαίσιο της γλώσσας προγραμματισμού C++, αλλά μπορούν 
να εφαρμοστούν και σε άλλες γλώσσες αντικειμενοστραφούς προγραμματισμού όπως η Java. Τα μοτίβα που πρότειναν οι GoF, 
χωρίζονται σε τρεις κατηγορίες, 
\begin{itemize}
    \item Δημιουργικά πρότυπα, για την δημιουργία αντικειμένων
    \item Δομικά πρότυπα, για την δημιουργία σχέσεων μεταξύ αντικειμένων
    \item Πρότυπα συμπεριφοράς, για τον καθορισμό του τρόπου αλληλεπίδρασης μεταξύ αντικειμένων.
\end{itemize}