\chapter{Έλεγχος}
\label{ch:testing}
Στο κεφάλαιο αυτό θα αναλυθούν οι έλεγχοι που υλοποιήθηκαν για τον κώδικα της εφαρμογής. 
Ο έλεγχος χωρίστηκε σε δύο επιμέρους κατηγορίες, στον έλεγχο του εργαλείου και στον έλεγχο των μοτίβων που υλοποιεί το εργαλείο.
Για την υλοποίησή τους χρησιμοποιήθηκε η βιβλιοθήκη Junit 4.
\section{Έλεγχος δομής σχεδιαστικών μοτίβων}
\label{sec:patternTesting}
\section{Έλεγχος μεθόδων}
\label{sec:patternManagerTesting}
Σε αυτήν την ενότητα, θα παρουσιαστούν οι έλεγχοι για την δομή των μοτίβων τα οποία περιγράφονται σε ένα αρχείο xml.
Για να επιβεβαιώσουμε την σωστή δομή κάθε μοτίβου, το μόνο που χρειάζεται είναι να ελέγξουμε την κλάση η οποία δημιουργεί 
τα απαραίτητα αντικείμενα αντλώντας δεδομένα από το αρχείο περιγραφής των μοτίβων, κλάση αυτή είναι η PatternManager.
Έχουμε δημιουργήσει τεστ τα οποία ελέγχουν εάν οι κλάσεις κάθε μοτίβου, έχουν το σωστό όνομα, υλοποιούν ή επεκτείνουν την σωστή 
διεπαφή ή κλάση αντίστοιχα και περιέχουν τις σωστές μεθόδους και πεδία. Επίσης υπάρχουν τεστ τα οποία επιβεβαιώνουν την δομή των 
διεπαφών, ελέγχοντας το όνομα και τις μεθόδους κάθε διεπαφής. Οι περιπτώσεις ελέγχων χωρίστηκαν σε δύο πακέτα, 
στο ένα πακέτο υπάρχουν οι περιπτώσεις ελέγχου των κλάσεων κάθε μοτίβου, και στο άλλο πακέτο υπάρχουν οι περιπτώσεις ελέγχου 
των διεπαφών κάθε μοτίβου. Πιο συγκεκριμένα, οι περιπτώσεις ελέγχου κατηγοριοποιούνται σύμφωνα με την 
κατηγορία των μοτίβων που έχουν προτείνει οι GoF. Στο πακέτο dpb.patternManagerTests.getClassTests, υπάρχουν οι εξής περιπτώσεις ελέγχου:
\begin{itemize}
    \item TestGetClassCreational, Τα τεστ αυτής της κλάσης ελέγχουν όλα τα μοτίβα της κατηγορίας creational
    \item TestGetClassStructural, Τα τεστ αυτής της κλάσης ελέγχουν όλα τα μοτίβα της κατηγορίας structural
    \item TestGetClassBehavioral, Τα τεστ αυτής της κλάσης ελέγχουν όλα τα μοτίβα της κατηγορίας behavioral
\end{itemize}
Για να επιβεβαιώσουμε την δομή κάθε μοτίβου, καλούμε την μέθοδο getClass(String, String), η οποία μας επιστρέφει 
μία λίστα με τις κλάσεις του συγκεκριμένου μοτίβου, για κάθε κλάση ελέγχουμε το όνομα της και αν υλοποιεί κάποια διεπαφή, 
το όνομα της διεπαφής, αντίστοιχα εάν επεκτείνει κάποια άλλη κλάση. Στη συνέχεια ελέγχουμε για κάθε πεδίο το όνομα του και 
τον τύπο του. Τέλος για κάθε μέθοδο ελέγχουμε το όνομα της, τον επιστρεφόμενο τύπο, 
το όνομα και τον τύπο κάθε παραμέτρου και το σώμα της μεθόδου εάν υπάρχει.
\linebreak
Στο πακέτο dpb.patternManagerTests.getInterfaceTests, υπάρχουν οι εξής περιπτώσεις ελέγχου:
\begin{itemize}
    \item TestGetInterfaceCreational, Τα τεστ αυτής της κλάσης ελέγχουν όλα τα μοτίβα της κατηγορίας creational
    \item TestGetInterfaceStructural, Τα τεστ αυτής της κλάσης ελέγχουν όλα τα μοτίβα της κατηγορίας structural
    \item TestGetInterfaceBehavioral, Τα τεστ αυτής της κλάσης ελέγχουν όλα τα μοτίβα της κατηγορίας behavioral
\end{itemize}
Για να επιβεβαιώσουμε την δομή κάθε μοτίβου, καλούμε την μέθοδο getClass(String, String), η οποία μας επιστρέφει 
μία λίστα με τις κλάσεις του συγκεκριμένου μοτίβου και στην συνέχεια την μέθοδο getInterfaces(), για κάθε διεπαφή 
ελέγχουμε το όνομα της. Στη συνέχεια για κάθε μέθοδο ελέγχουμε το όνομα της, τον επιστρεφόμενο τύπο και
το όνομα και τον τύπο κάθε παραμέτρου.
\section{Έλεγχος μεθόδων δημιουργίας πηγαίου κώδικα java}
\label{sec:patternGeneratorTesting}