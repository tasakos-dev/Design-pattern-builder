\chapter{Κάρτες αρμοδιοτήτων και συνεργασιών κλάσεων}
\label{app:crc}

\begin{table}[H]
    \centering
    \begin{tabular}{|p{5cm}|p{5cm}|}
        \hline
        \multicolumn{2}{|l|}{Όνομα κλάσης: PatternClass} \\
        \hline
        \textbf{Αρμοδιότητες} & \textbf{Συνεργασίες} \\
        \hline
        \begin{itemize}
            \item Αυτή η κλάση είναι υπεύθυνη για την μοντελοποίηση μίας κλάσης ενός μοτίβου.
        \end{itemize} &   
        \begin{itemize}
            \item Method
            \item Field
        \end{itemize} \\
        \hline
    \end{tabular}
    \caption{PatternClass κάρτα αρμοδιοτήτων.}
    \label{tab:PatternClassCRC}
\end{table}
\begin{table}[H]
    \centering
    \begin{tabular}{|p{5cm}|p{5cm}|}
        \hline
        \multicolumn{2}{|l|}{Όνομα κλάσης: PatternInterface} \\
        \hline
        \textbf{Αρμοδιότητες} & \textbf{Συνεργασίες} \\
        \hline
        \begin{itemize}
            \item Αυτή η κλάση είναι υπεύθυνη για την μοντελοποίηση μίας διεπαφής ενός μοτίβου.
        \end{itemize} &   
        \begin{itemize}
            \item Method
        \end{itemize} \\
        \hline
    \end{tabular}
    \caption{PatternInterface κάρτα αρμοδιοτήτων.}
    \label{tab:PatternInterfaceCRC}
\end{table}
\begin{table}[H]
    \centering
    \begin{tabular}{|p{5cm}|p{5cm}|}
        \hline
        \multicolumn{2}{|l|}{Όνομα κλάσης: Method} \\
        \hline
        \textbf{Αρμοδιότητες} & \textbf{Συνεργασίες} \\
        \hline
        \begin{itemize}
            \item Η κλάση αυτή διατηρεί τα χαρακτηριστικά μίας μεθόδου που ανήκει σε κάποια κλάση ή διεπαφή.
        \end{itemize} &    
        % \begin{itemize}
        %     \item hi
        % \end{itemize} 
        \\
        \hline
    \end{tabular}
    \caption{Method κάρτα αρμοδιοτήτων.}
    \label{tab:MethodCRC}
\end{table}
\begin{table}[H]
    \centering
    \begin{tabular}{|p{5cm}|p{5cm}|}
        \hline
        \multicolumn{2}{|l|}{Όνομα κλάσης: Field} \\
        \hline
        \textbf{Αρμοδιότητες} & \textbf{Συνεργασίες} \\
        \hline
        \begin{itemize}
            \item Η κλάση αυτή διατηρεί τα χαρακτηριστικά ενός πεδίου που ανήκει σε κάποια κλάση.
        \end{itemize} &   
        % \begin{itemize}
        %     \item hi
        % \end{itemize} 
        \\
        \hline
    \end{tabular}
    \caption{Field κάρτα αρμοδιοτήτων.}
    \label{tab:FieldCRC}
\end{table}
\begin{table}[H]
    \centering
    \begin{tabular}{|p{10cm}|p{5cm}|}
        \hline
        \multicolumn{2}{|l|}{Όνομα κλάσης: FileParser} \\
        \hline
        \textbf{Αρμοδιότητες} & \textbf{Συνεργασίες} \\
        \hline
        \begin{itemize}
            \item Η κλάση αυτή είναι υπεύθυνη για την άντληση των κατηγοριών που ορίζουν οι GoF \cite{GoF}.
            \item Η κλάση αυτή αρμόδια για την άντληση των μοτίβων κάποιας κατηγορίας.
            \item Η κλάση αυτή είναι αρμόδια για την άντληση των κλάσεων ενός μοτίβου, όπως και την ορατότητα της κλάσης.
            \item Η κλάση αυτή είναι αρμόδια για την άντληση των διεπαφών ενός μοτίβου όπως και την ορατότητα της διεπαφής.
            \item Η κλάση αυτή είναι αρμόδια για την άντληση των μεθόδων κάποιας κλάσης, καθώς τους διάφορους τροποποιητές που μπορεί 
            να έχει μία μέθοδος, όπως και τον επιστρεφόμενο τύπο της μεθόδου και τέλος τον κώδικα της μεθόδου.
            \item Η κλάση αυτή είναι αρμόδια για την άντληση των πεδίων κάποιας κλάσης ενός μοτίβου.
            \item Η κλάση αυτή είναι αρμόδια για την άντληση των μεθόδων κάποιας διεπαφής μαζί με τους 
            τροποποιητές της μεθόδου και τον επιστρεφόμενο τύπο της.
            \item Η κλάση αυτή είναι αρμόδια για την άντληση των annotations ενός στοιχείου του μοτίβου.
        \end{itemize} &   
        % \begin{itemize}
        %     \item 
        % \end{itemize}
         \\
        \hline
    \end{tabular}
    \caption{FileParser κάρτα αρμοδιοτήτων.}
    \label{tab:fileParserCRC}
\end{table}
\begin{table}[H]
    \centering
    \begin{tabular}{|p{10cm}|p{5cm}|}
        \hline
        \multicolumn{2}{|l|}{Όνομα κλάσης: PatternManager} \\
        \hline
        \textbf{Αρμοδιότητες} & \textbf{Συνεργασίες} \\
        \hline
        \begin{itemize}
            \item Η κλάση αυτή είναι αρμόδια για την διάθεση των κατηγοριών των μοτίβων στην γραφική διεπαφή.
            \item Η κλάση αυτή είναι αρμόδια για την διάθεση των μοτίβων κάποιας κατηγορίας στην γραφική διεπαφή.
            \item Η κλάση αυτή είναι αρμόδια για την δημιουργία αντικειμένων τύπου PatternClass και την διαθέση τους στο front-end τμήμα του 
            εργαλείου, ώστε να μπορεί να διαχειριστεί ο χρήστης τις διάφορες κλάσεις ενός μοτίβου.
            \item Η κλάση αυτή είναι αρμόδια για την δημιουργία αντικειμένων τύπου PatternInterface και την διαθέση τους στο front-end τμήμα του 
            εργαλείου, ώστε να μπορεί να διαχειριστεί ο χρήστης τις διάφορες διεπαφές ενός μοτίβου.
            \item Η κλάση αυτή είναι αρμόδια για την ενημέρωση του ονόματος μίας κλάσης.
            \item Η κλάση αυτή είναι αρμόδια για την ενημέρωση του ονόματος μίας διεπαφής.
            \item Η κλάση αυτή είναι αρμόδια για την ενημέρωση του ονόματος μίας μεθόδου.
            \item Η κλάση αυτή είναι αρμόδια για την ενημέρωση του ονόματος ενός πεδίου.
        \end{itemize} &   
        \begin{itemize}
            \item FileParser
            \item PatternClass
            \item PatternInterface
            \item Method
            \item Field
            \item Property
        \end{itemize} \\
        \hline
    \end{tabular}
    \caption{PatternManager κάρτα αρμοδιοτήτων.}
    \label{tab:PatternManagerCRC}
\end{table}
\begin{table}[H]
    \centering
    \begin{tabular}{|p{5cm}|p{5cm}|}
        \hline
        \multicolumn{2}{|l|}{Όνομα κλάσης: ClassGenerator} \\
        \hline
        \textbf{Αρμοδιότητες} & \textbf{Συνεργασίες} \\
        \hline
        \begin{itemize}
            \item Η κλάση αυτή είναι αρμόδια για την προσθήκη των annotations στο classpath του έργου που έχει επιλέξει ο προγραμματιστής.
            \item Η κλάση αυτή είναι αρμόδια για την παραγωγή των πηγαίων αρχείων java μίας κλάσης.
        \end{itemize} &   
        \begin{itemize}
            \item PatternManager
            \item PatternClass
            \item PatternInterface
            \item Method
            \item Field
            \item Property
        \end{itemize} \\
        \hline
    \end{tabular}
    \label{tab:ClassGeneratorCRC}
    \caption{ClassGenerator κάρτα αρμοδιοτήτων}
\end{table}
\begin{table}[H]
    \centering
    \begin{tabular}{|p{5cm}|p{5cm}|}
        \hline
        \multicolumn{2}{|l|}{Όνομα κλάσης: InterfaceGenerator} \\
        \hline
        \textbf{Αρμοδιότητες} & \textbf{Συνεργασίες} \\
        \hline
        \begin{itemize}
            \item Η κλάση αυτή είναι αρμόδια για την προσθήκη των annotations στο classpath του έργου που έχει επιλέξει ο προγραμματιστής.
            \item Η κλάση αυτή είναι αρμόδια για την παραγωγή των πηγαίων αρχείων java μίας διεπαφής.
        \end{itemize} &   
        \begin{itemize}
            \item PatternManager
            \item PatternClass
            \item PatternInterface
            \item Method
            \item Field
            \item Property
        \end{itemize} \\
        \hline
    \end{tabular}
    \caption{InterfaceGenerator κάρτα αρμοδιοτήτων.}
    \label{tab:InterfaceGeneratorCRC}
\end{table}