\chapter{Επίλογος}
\label{ch:epilogue}
\section{Σύνοψη και συμπεράσματα}
\label{sec:conclusion}
Στη διπλωματική αυτή δημιουργήθηκε ένα καινούριο εργαλείο 
με βασική λειτουργικότητα την ενσωμάτωση σχεδιαστικών μοτίβων σε πηγαίο κώδικα Java. 
Το εργαλείο βασίστηκε στα ήδη υπάρχοντα εργαλεία, Pattern Wizard, Patternbox, Design Pattern Automation Toolkit, καθώς και 
Alphaworks Design Pattern Toolkit \cite{PatternBox}, εξαλείφοντας  και συνεισφέροντας  σε βασικές 
τους ελλείψεις όπως η πληθώρα μοτίβων, την παραμετροποίηση των μοτίβων από τον χρήστη και τέλος, τον χαρακτηρισμό 
κάθε παραγόμενης κλάσης και διεπαφής, με την χρήση των annotations.\par
Επιπλέον, κάνοντας μία περιήγηση στο εργαλείο παρατηρούμε τα εξής: όλα τοποθετημένα στο ίδιο παράθυρο με τον κώδικα του χρήστη, 
καθώς πρόκειται για ένα εργαλείο το οποίο είναι επέκταση του περιβάλλοντος Eclipse.
Χάρη στον σχεδιασμό του, μπορεί κάποιος πολύ εύκολα να επεκτείνει τα διαθέσιμα μοτίβα 
και με άλλα μοτίβα εκτός από αυτά των GoF \cite{GoF}.\par Τέλος, η δυνατότητα των χρηστών να ελέγχουν ανά πάσα ώρα και στιγμή την πληροφορία 
που βρίσκεται μέσα στο εργαλείο. Καθώς, οι προσθήκες νέων μοτίβων από τον υπολογιστή του χρήστη όπως και η διαγραφή τους επιτρέπει, 
την εξατομικευμένη λειτουργία του. Έτσι, καλύπτονται πλήρως οι ανάγκες ενός προγραμματιστή ο οποίος έχει καλή γνώση των σχεδιαστικών μοτίβων.
\section{Μελλοντικές επεκτάσεις}
\label{sec:features}
Στην ενότητα, αυτή θα προταθούν μερικές επεκτάσεις του εργαλείου. Στο μέλλον, θα μπορούσε να προστεθεί η δυνατότητα 
να μπορεί κάποιος προγραμματιστής να αντιστοιχίσει κλάσεις και διεπαφές του μοτίβου με υπάρχουσες, 
οι οποίες βρίσκονται στο τρέχον έργο του προγραμματιστή. Έτσι, μπορεί να προσαρμόσει 
τις ήδη υπάρχουσες κλάσεις και διεπαφές στις ανάγκες του μοτίβου. Με αυτό τον τρόπο, δίνεται η δυνατότητα στον χρήστη
να εφαρμόσει διάφορα μοτίβα σε έργα τα οποία συντηρεί, αναδομώντας τα έργα αυτά και βελτιώνοντας τα.