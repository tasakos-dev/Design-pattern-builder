\chapter{Επίλογος}
\label{chepilogue}
\section{Σύνοψη και συμπεράσματα}
\label{secconclusion}
Στη Διπλωματική αυτή δημιουργήθηκε ένα καινούριο εργαλείο 
με βασική λειτουργικότητα την ενσωμάτωση σχεδιαστικών μοτίβων σε πηγαίο κώδικα java. 
Το εργαλείο βασίστηκε στα ήδη υπάρχοντα εργαλεία, Pattern Wizard, Patternbox, Design Pattern Automation Toolkit, καθώς και 
Alphaworks Design Pattern Toolkit \cite{PatternBox}, εξαλείφοντας  και συνεισφέροντας  σε βασικές 
τους ελλείψεις όπως η πληθώρα μοτίβων, την παραμετροποίηση των μοτίβων από τον χρήστη και τέλος, τον χαρακτηρισμό 
κάθε παραγόμενης κλάσης και διεπαφής, με την χρήση των annotations. Επιπλέον, η περιήγηση στο εργαλείο είναι ιδιαίτερα ξεκούραστη, 
με εύκολα μενού. Όλα τοποθετημένα στο ίδιο παράθυρο με τον κώδικα του χρήστη, καθώς πρόκειται για ένα εργαλείο το οποίο 
είναι επέκταση του περιβάλλοντος Eclipse.
Χάρη στον σχεδιασμό του, μπορεί κάποιος πολύ εύκολα να επεκτείνει τα διαθέσιμα μοτίβα 
και με άλλα μοτίβα εκτός από αυτά των GoF \cite{GoF}. Τέλος η δυνατότητα των χρηστών να ελέγχουν ανά πάσα ώρα και στιγμή 
την πληροφορία που βρίσκεται μέσα στο εργαλείο, καθώς και οι προσθήκες  νέων μοτίβων από τον υπολογιστή του χρήστη καθώς 
και η διαγραφή τους επιτρέπει την εξατομικευμένη λειτουργία του καλύπτοντας πλήρως τις ανάγκες 
προγραμματιστή ο οποίος έχει καλή γνώση των σχεδιαστικών μοτίβων.
\section{Μελλοντικές επεκτάσεις}
\label{secfeatures}
Στην ενότητα αυτή θα προταθούν μερικές επεκτάσεις του εργαλείου. Στο μέλλον θα μπορούσε να προστεθεί η δυνατότητα, 
να μπορεί κάποιος προγραμματιστής να αντιστοιχίσει, κλάσεις και διεπαφές με υπάρχουσες κλάσεις και διεπαφές, 
οι οποίες βρίσκονται στο τρέχων έργο του προγραμματιστή, έτσι ώστε να μπορεί ο προγραμματιστής να προσαρμόσει 
τις ήδη υπάρχουσες κλάσεις και διεπαφές, στις ανάγκες του μοτίβου. Έτσι δίνεται η δυνατότητα, 
σε κάποιον προγραμματιστή να μπορεί εφαρμόσει διάφορα μοτίβα σε έργα τα οποία συντηρεί, 
αναδομώντας τα έργα αυτά και βελτιώνοντας τα.
