\chapter{Εισαγωγή}
\label{ch:Introduction}
\section{Σχεδιαστικά μοτίβα}
\label{sec:patterns}
Ένα σχεδιαστικό μοτίβο \cite{GoF}, ορίζει μια γενική λύση σε ένα συχνά εμφανιζόμενο πρόβλημα, γραμμένο
ως πρότυπο ή ως σύνολο σχέσεων. Στην επιστήμη των υπολογιστών, αυτό μεταφράζεται σε ένα
προγραμματιστικό μοτίβο ή ένα διάγραμμα αντικειμένων που αναπαριστά μια 
γνωστή λύση σε μια κοινή εργασία. \par
Όπως οι αλγόριθμοι, έτσι και τα πρότυπα σχεδίασης είναι μια δοκιμασμένη
και αποδεκτή λύση και ως εκ τούτου, οι άνθρωποι τα αναπτύσσουν και 
τα προσαρμόζουν αντί να τα ανακαλύπτουν.
Και τα δύο αντιπροσωπεύουν μια προσέγγιση ενός προβλήματος περισσότερο από 
την ειδική υλοποίηση της λύσης. Ωστόσο, διαφέρουν ως προς τον σκοπό τους 
οι αλγόριθμοι βελτιστοποιούν το κόστος μιας λύσης, 
ενώ τα πρότυπα σχεδίασης βελτιστοποιούν τη σαφήνειά της.
\section{Στόχοι}
\label{sec:Objectives}
Αυτή η διπλωματική επιδιώκει την ανάπτυξη ενός εργαλείου, το Design Pattern Builder, 
που επιτρέπει που επιτρέπει σε αρχάριους προγραμματιστές να εισάγουν 
σχεδιαστικά μοτίβα στον κώδικά τους σε Java με ένα απλό βήμα. 
Το Design Pattern Builder, παρέχει μια διεπαφή όπου ο χρήστης μπορεί να επιλέξει το επιθυμητό 
σχεδιαστικό μοτίβο, να κάνει τις απαραίτητες ρυθμίσεις 
και να εισάγει αυτόματα τον αντίστοιχο κώδικα στον δικό του κώδικα. 
Το εργαλείο περιλαμβάνει μια συλλογή από τα σχεδιαστικά μοτίβα που 
προτείνονται από την ομάδα των GOF \cite{GoF}, 
όπως το Singleton, το Factory, το Observer και άλλα. 
Ο χρήστης έχει τη δυνατότητα να εξερευνήσει αυτήν τη συλλογή, να επιλέξει 
το επιθυμητό μοτίβο και να εισάγει αυτόματα τον απαιτούμενο κώδικα στον κώδικά 
του. Ο στόχος είναι να διευκολυνθεί ο αρχάριος προγραμματιστής στη χρήση 
σχεδιαστικών μοτίβων και να ενισχυθεί η προγραμματιστική του απόδοση και ακρίβεια, 
επιτρέποντάς του να εισάγει εύκολα τα απαραίτητα σχεδιαστικά μοτίβα στον κώδικά 
του, χωρίς την ανάγκη για χειροκίνητη υλοποίηση. 
Συνολικά, ο στόχος είναι να δημιουργηθεί ένα εύχρηστο και βοηθητικό 
εργαλείο που θα επιτρέπει στους αρχάριους προγραμματιστές να αξιοποιούν 
τα σχεδιαστικά μοτίβα.
\section{Δομή της διπλωματικής εργασίας}
\label{sec:Structure}
Το υπόλοιπο της  διπλωματικής  εργασίας  περιλαμβάνει τα εξής κεφάλαια:
Στο κεφάλαιο \ref{ch:relativeWork}, γίνεται ανάλυση της παρελθοντικής δουλειάς που έχει γίνει και είναι σχετική με
το εργαλείο που πραγματεύεται η παρούσα διπλωματική. Το κεφάλαιο \ref{ch:requirmentAnalysis}, περιέχει τις ιστορίες χρήστη
καθώς, και τις περιπτώσεις χρήσης του εργαλείου. 
Το κεφάλαιο \ref{ch:architecture}, εξετάζει την αρχιτεκτονική του εργαλείου, 
καθώς παρουσιάζονται λεπτομερώς οι κλάσεις και τα πακέτα που αποτελούν το εργαλείο. Στο κεφάλαιο \ref{ch:testing}, περιέχονται 
λεπτομέρειες σχετικά με τον αυτοματοποιημένο έλεγχο του εργαλείου. Στο κεφάλαιο \ref{ch:manual}, παρουσιάζονται 
οι οδηγίες χρήσης του εργαλείου. Τέλος, στο κεφάλαιο \ref{ch:epilogue} παρουσιάζονται οι μελλοντικές επεκτάσεις του εργαλείου.