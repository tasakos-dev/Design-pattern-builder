%
% Περάστε στο πακέτο cseuoi-thesis την επιλογή για τη γλώσσα συγγραφής:
% "en" ή "gr" για Αγγλικά ή Ελληνικά αντίστοιχα.
%
\documentclass[gr]{cseuoi-thesis} % Διπλωματική εργασία (στα Ελληνικά)


% 
% Συμπληρώστε τα στοιχεία σας στις παρακάτω εντολές (αφαιρώντας το \colorbox{gray}{})
%
\titleGr{Αυτοματοποιημένη ενσωμάτωση μοτίβων σχεδίασης σε πηγαίο κώδικα Java}
\titleEn{Automated incorporation of design patterns in Java source code}
\authorGr{Αναστάσιος Λιόντος}
\authorEn{Anastasios Liontos}
\aitiatiki{Αναστάσιε Λιόντο}
\dateGr{Ιούλιος 2023}
\dateEn{July 2023}
\advisorGr{Απόστολος Ζάρρας, Καθηγητής}
\advisorEn{Apostolos Zarras, Professor}


% Πακέτο για την εμφάνιση περιθωρίων (χρήσιμο για την εύρεση overfull boxes)
% \usepackage{showframe}

% Πακέτο για τη διατήρηση των floats (εικόνες κ.α.) εντός των ενοτήτων
%\usepackage[section]{placeins}

\usepackage{hyperref}
\usepackage{totcount}
\usepackage{pifont}
\usepackage{color}
\usepackage{listings}
\usepackage{caption}


\definecolor{gray}{rgb}{0.4,0.4,0.4}
\definecolor{darkblue}{rgb}{0.0,0.0,0.6}
\definecolor{cyan}{rgb}{0.0,0.6,0.6}

\DeclareCaptionFont{white}{\color{white}}
\DeclareCaptionFormat{listing}{\colorbox[cmyk]{0.43, 0.35, 0.35, 0.01}{\parbox{\dimexpr\linewidth-2\fboxsep\relax}{\hspace{15pt}#1#2#3}}}
\captionsetup[lstlisting]{format=listing,labelfont=white,textfont=white, singlelinecheck=false, margin=0pt, font={bf,footnotesize}}

\renewcommand\lstlistingname{Παράδειγμα Κώδικα}
\renewcommand\lstlistlistingname{Κατάλογος Παραδειγμάτων Κώδικα}
\newcommand{\anotelia}{\char"02D9}

\lstdefinelanguage{XML}
{
  numbers=left,
  basicstyle=\ttfamily\color{darkblue}\bfseries,
  morestring=[b]",
  morestring=[s]{>}{<},
  morecomment=[s]{<?}{?>},
  stringstyle=\color{black},
  identifierstyle=\color{darkblue},
  keywordstyle=\color{cyan},
  morekeywords={xmlns,version,type}% list your attributes here
  numbersep=12pt,
	tabsize=4,
	extendedchars=true,
  breaklines=true,
  frame=b,
	xleftmargin=24pt,
	framexleftmargin=24pt,
	framexrightmargin=0pt
}

\lstset{language=XML,basicstyle=\ttfamily,breaklines=true}

\regtotcounter{chapter}

\begin{document}

% Σελίδες χωρίς αρίθμηση
\pagenumbering{gobble}

% Εκτύπωση της σελίδας τίτλου
\maketitle

% Αρχικοποίηση του minitoc
\dominitoc[n]

\chapter*{\cseafierwsi}

\vfill

\begin{flushright}
\emph{Στην Οικογένεια μου}
\end{flushright}

\vfill

\bigskip % Προαιρετικό
\include{FrontMatter/Acknowledgements} % Προαιρετικό


% Σελίδες με αρίθμηση i, ii, iii, iv, ...
\pagenumbering{roman}

% Περιεχόμενα
\pdfbookmark{\contentsname}{contents} % hyperref
\tableofcontents

% Κατάλογος Σχημάτων
\addstarredchapterc{\listfigurename} % minitoc
\listoffigures

% Κατάλογος Πινάκων
\addstarredchapterc{\listtablename} % minitoc
\listoftables

% Κατάλογος Παραδειγμάτων κώδικα
\addstarredchapterc{\lstlistingname} % minitoc
\lstlistoflistings

% Κατάλογος Αλγορίθμων
%\addstarredchapterc{\listalgorithmname} % minitoc
%\listof{algorithm}{\listalgorithmname}

%\chapter*{\glossaryname}
% Εισαγωγή του κεφάλαιου στα περιεχόμενα
\addstarredchapter{\glossaryname} %minitoc
 % Προαιρετικό

% Περίληψη στη γλώσσα συγγραφής (π.χ. Ελληνικά)
\chapter*{\abstractname}
\addstarredchapter{\abstractname} % minitoc
\makecseabstract

Το εργαλείο αυτό αποτελεί μία επέκταση του Eclipse IDE, έχει ως στόχο να βοηθήσει κάποιον αρχάριο κυρίως, προγραμματιστή, 
να εισάγει εύκολα και γρήγορα κάποιο σχεδιαστικό μοτίβο. Μέσω μίας γραφικής διεπαφής
ένας προγραμματιστής μπορεί να επιλέξει κατηγορία και το μοτίβο
που επιθυμεί, να ρυθμίσει τις κλάσεις και τις διεπαφές σύμφωνα με
τις απαιτήσεις του κώδικα του και τέλος να εξάγει τον σκελετό του 
μοτίβου στον κώδικα του. 
\linebreak
\linebreak
\textbf{Λέξεις κλειδία:} Επέκταση Eclipse, 
διπλωματική εργασία, σχεδιαστικά μοτίβα, προγραμματιστής

% Περίληψη στην αντίθετη γλώσσα από τη γλώσσα συγγραφής (π.χ. Αγγλικά)
\chapter*{\csebilabstract}
\addstarredchapter{\csebilabstract} % minitoc
\makecsebilabstract




% Σελίδες με αρίθμηση 1, 2, 3, 4, ...
\pagenumbering{arabic}

% Εισαγωγή των κεφαλαίων
\chapter{Εισαγωγή}
\label{ch:Introduction}
\section{Σχεδιαστικά μοτίβα}
\label{sec:patterns}
Ένα σχεδιαστικό μοτίβο \cite{GoF} ορίζει μια γενική λύση σε ένα συχνά εμφανιζόμενο πρόβλημα, γραμμένο
ως πρότυπο ή ως σύνολο σχέσεων. Στην επιστήμη των υπολογιστών, αυτό μεταφράζεται σε ένα
προγραμματιστικό μοτίβο ή ένα διάγραμμα αντικειμένων που αναπαριστά μια 
γνωστή λύση σε ένα κοινό πρόβλημα. \par
Όπως οι αλγόριθμοι, έτσι και τα πρότυπα σχεδίασης είναι μια δοκιμασμένη
και αποδεκτή λύση και ως εκ τούτου, οι άνθρωποι τα αναπτύσσουν και 
τα προσαρμόζουν αντί να τα ανακαλύπτουν.
Και τα δύο αντιπροσωπεύουν μια προσέγγιση ενός προβλήματος περισσότερο από 
την ειδική υλοποίηση της λύσης. Ωστόσο, διαφέρουν ως προς τον σκοπό τους 
οι αλγόριθμοι βελτιστοποιούν το κόστος μιας λύσης, 
ενώ τα πρότυπα σχεδίασης βελτιστοποιούν τη σαφήνειά της.
\section{Στόχοι}
\label{sec:Objectives}
Η παρούσα διπλωματική εργασία επιδιώκει την ανάπτυξη ενός εργαλείου, το οποίο ονομάζεται Design Pattern Builder. 
Το εργαλείο αυτό επιτρέπει σε αρχάριους προγραμματιστές να εισάγουν 
σχεδιαστικά μοτίβα στον κώδικά τους σε Java με απλά βήματα. 
Με το Design Pattern Builder ο χρήστης έχει την δυνατότητα: να επιλέξει το επιθυμητό 
σχεδιαστικό μοτίβο, να κάνει τις απαραίτητες ρυθμίσεις 
και να εισάγει αυτόματα τον αντίστοιχο κώδικα στον δικό του κώδικα. 
Το εργαλείο περιλαμβάνει μια συλλογή από τα σχεδιαστικά μοτίβα που 
προτείνονται από την ομάδα των GOF \cite{GoF}, 
όπως το Singleton, το Factory, το Observer και άλλα. 
Ο χρήστης έχει τη δυνατότητα να εξερευνήσει αυτήν τη συλλογή, να επιλέξει 
το επιθυμητό μοτίβο και να εισάγει αυτόματα τον απαιτούμενο κώδικα στον κώδικά 
του. Ο στόχος είναι να διευκολυνθεί ο αρχάριος προγραμματιστής στη χρήση 
σχεδιαστικών μοτίβων και να ενισχυθεί η προγραμματιστική του απόδοση και ακρίβεια, 
επιτρέποντάς του να εισάγει εύκολα τα απαραίτητα σχεδιαστικά μοτίβα στον κώδικά 
του, χωρίς την ανάγκη για χειροκίνητη υλοποίηση. 
Συνολικά, ο στόχος είναι να δημιουργηθεί ένα εύχρηστο και βοηθητικό 
εργαλείο που θα επιτρέπει στους αρχάριους προγραμματιστές να αξιοποιούν 
τα σχεδιαστικά μοτίβα.
\section{Δομή της διπλωματικής εργασίας}
\label{sec:Structure}
Το υπόλοιπο της  διπλωματικής  εργασίας  περιλαμβάνει τα εξής κεφάλαια:
Στο κεφάλαιο \ref{ch:relativeWork}, γίνεται ανάλυση της παρελθοντικής δουλειάς που έχει γίνει και είναι σχετική με
το εργαλείο που πραγματεύεται η παρούσα διπλωματική. Το κεφάλαιο \ref{ch:requirmentAnalysis}, περιέχει τις ιστορίες χρήστη
καθώς, και τις περιπτώσεις χρήσης του εργαλείου. 
Το κεφάλαιο \ref{ch:architecture}, εξετάζει την αρχιτεκτονική του εργαλείου, 
καθώς παρουσιάζονται λεπτομερώς οι κλάσεις και τα πακέτα που αποτελούν το εργαλείο. Στο κεφάλαιο \ref{ch:testing}, περιέχονται 
λεπτομέρειες σχετικά με τον αυτοματοποιημένο έλεγχο του εργαλείου. Στο κεφάλαιο \ref{ch:manual}, παρουσιάζονται 
οι οδηγίες χρήσης του εργαλείου. Τέλος, στο κεφάλαιο \ref{ch:epilogue} παρουσιάζονται οι μελλοντικές επεκτάσεις του εργαλείου.
\chapter{Σχετική Δουλειά}
\label{ch:relativeWork}
\section{Σχετικά εργαλεία}
\label{sec:relativeTools}
\subsection{Pattern Wizard}
\label{subsec:patternWizard}
Το εργαλείο\textsubscript{\cite{PatternBox}} αυτό, προσφέρει ως λειτουργίες, την επιλογή κάποιου μοτίβου 
από τα Adapter, Abstract Factory και Observer, καθώς και την επιλογή γλώσσας προγραμματισμού, μέχρι στιγμής την Java. Στην συνέχεια 
ο χρήστης έχει την δυνατότητα να αντιστοιχίσει υπάρχων κώδικά στο μοτίβο που έχει επιλέξει και το εργαλείο μετατρέπει τον κώδικα 
σε κώδικα που είναι συμβατός με το μοτίβο. Τέλος, ο προγραμματιστής έχει την δυνατότητα να επιλέξει κάποιο κενό πηγαίο αρχείο και 
το εργαλείο εξάγει τον σκελετό του μοτίβου.
\subsection{Patternbox}
\label{subsec:Patternbox}
Το εργαλείο\textsubscript{\cite{PatternBox}} αυτό υποστηρίζει 16 μόνο μοτίβα από αυτά που έχουν προτείνει οι GoF\textsubscript{\cite{GoF}} δημιουργεί ένα ειδικό 
αρχείο το οποίο αναπαριστά το μοτίβο. Μέσα από το αρχείο μπορεί ο προγραμματιστής να επιλέξει κάποιο στοιχείο του μοτίβου και 
να επιλέξει την τοποθεσία οου θα εξαχθεί είτε η κλάση είτε η διεπαφή που έχει επιλέξει ο χρήστης. 
Αφού ο χρήστης ολοκληρώσει την διαδικασία, το εργαλείο παράγει απλώς τον σκελετό του μοτίβου, χωρίς να τροποποιεί τον κώδικα.
\subsection{Design Pattern Automation Toolkit}
\label{subsec:dpa}
Το εργαλείο\textsubscript{\cite{PatternBox}} αυτό υποστηρίζει και τα 23 μοτίβα των GoF\textsubscript{\cite{GoF}}, υποστηρίζει λειτουργίες 
όπως επιλογή μοτίβου και γλώσσας προγραμματισμού. Επίσης κάποιος προγραμματιστής μπορεί να εισάγει νέα μοτίβα που μπορεί να υλοποιεί το εργαλείο.
Τέλος δεν τροποποιεί τον υπάρχων κώδικά και παρά μόνο εξάγει τον σκελετό του μοτίβου.
\subsection{Alphaworks Design Pattern Toolkit}
\label{subsec:ALPHAWORKS}
Αυτό το εργαλείο\textsubscript{\cite{PatternBox}} είναι αντίστοιχο του Patternbox\textsubscript{\ref{sec:Patternbox}}, με την διαφορά ότι το Alphaworks μετασχηματίζει τον κώδικα
του προγραμματιστή σε ένα ενδιάμεσο αρχείο τύπου xml, αφού ο προγραμματιστής έχει εισάγει στον κώδικά του κατάλληλα tags.
Στη συνέχεια ο προγραμματιστής κάνει τις αλλαγές που επιθυμεί στο αρχείο xml και το εργαλείο μετασχηματίζει πίσω σε κώδικα συμβατό με το μοτίβο.
\subsection{Σύγκριση}
\label{subsec:compare}
Το Design Pattern Builder, το οποίο είναι το εργαλείο που αποτελεί αντικείμενο της διπλωματικής αυτής, παρουσιάζει πλεονεκτήματα έναντι των παραπάνω εργαλείων,
καθώς ο προγραμματιστής μπορεί να τροποποιήσει οποιοδήποτε μοτίβο, ώστε το μοτίβο να ταιριάξει με τις ανάγκες του έργου του προγραμματιστή. 
Επίσης εξάγει annotations -τα οποία αποδίδουν τον ρόλο που έχει η κλάση ή διεπαφή-  στον σκελετό του μοτίβου ώστε 
να διευκολύνει τον προγραμματιστή στην εφαρμογή του μοτίβου. Τέλος το εργαλείο που προτείνουμε προσφέρει και τα 23 μοτίβα 
των GoF\textsubscript{\cite{GoF}}.
\linebreak
\linebreak
Από την άλλη, το εργαλείο που προτείνουμε υστερεί ως προς την δυνατότητα αντιστοίχισης 
ενός στοιχείου του μοτίβου με κάποιο στοιχείο του έργου του προγραμματιστή. Τέλος ένα ακόμα μειονέκτημα 
είναι η δυνατότητα εξαγωγής κώδικα και σε άλλες γλώσσες προγραμματισμού εκτός της java.
\section{Υπόβαθρο}
\label{sec:background}
Τα σχεδιαστικά μοτίβα GoF (Gang of Four), είναι ένα σύνολο σχεδιαστικών προτύπων που περιγράφονται στο βιβλίο \cite{GoF}.
Αυτά τα πρότυπα προέκυψαν από την εμπειρία και την εμπειρογνωμοσύνη των τεσσάρων συγγραφέων, οι GoF, και αποτελούν γενικές λύσεις 
για συχνά προβλήματα στον σχεδιασμό λογισμικού. Οι GoF πρότειναν 23 μοτίβα στο πλαίσιο της γλώσσας προγραμματισμού C++, αλλά μπορούν 
να εφαρμοστούν και σε άλλες γλώσσες αντικειμενοστραφούς προγραμματισμού όπως η Java. Τα μοτίβα που πρότειναν οι GoF, 
χωρίζονται σε τρεις κατηγορίες, 
\begin{itemize}
    \item Δημιουργικά πρότυπα, για την δημιουργία αντικειμένων
    \item Δομικά πρότυπα, για την δημιουργία σχέσεων μεταξύ αντικειμένων
    \item Πρότυπα συμπεριφοράς, για τον καθορισμό του τρόπου αλληλεπίδρασης μεταξύ αντικειμένων.
\end{itemize}
\chapter{Ανάλυση Απαιτήσεων}
Στο κεφάλαιο αυτό θα δοθούν οι ιστορίες χρήστη που αφορούν το εργαλείο, σε μορφή πινάκων, καθώς  και οι περιπτώσεις χρήσης του.

\label{ch:requirmentAnalysis}
\section{Ιστορίες Χρήστη}
\label{sec:userStories}
Οι ιστορίες χρήστη αποτελούν  άτυπες περιγραφές  των χαρακτηριστικών του εργαλείου μας και των δυνατοτήτων του σε φυσική γλώσσα. Γράφονται από την πλευρά του χρήστη του εργαλείου σε μορφή καρτών.
\begin{table}[H]
	\hspace*{-0.2cm}
    \centering
    \scriptsize
    \caption{Ιστορίες χρήστη}
    \label{tab:userStories}
	\begin{tabular}{|p{1.5cm}|p{3.5cm}|p{4.5cm}|p{4.7cm}|}
    \hline
        \textbf{Ιστορία Χρήστη} & \textbf{Σαν [τύπος χρήστη]} & \textbf{Θέλω να [πραγματοποιήσω ένα έργο]} & \textbf{Ώστε να μπορώ [να πετύχω έναν στόχο]} \\ \hline     \hline
        ΙΧ1 & Προγραμματιστής & Να μπορώ να επιλέξω κατηγορία μοτίβων. & Έτσι ώστε να εισάγω αυτόματα στον κώδικά μου το σκελετό ενός μοτίβου της κατηγορίας αυτής. \\ \hline
        ΙΧ2 & Προγραμματιστής & Να μπορώ να επιλέγω ένα μοτίβο μιας κατηγορίας. & Έτσι ώστε να  εισάγω αυτόματα στον κώδικά μου το σκελετό του μοτίβου αυτού. \\ \hline
        ΙΧ3 & Προγραμματιστής & Να μπορώ να καθορίσω τα ονόματα των κλάσεων που θα δημιουργηθούν αυτόματα. & Έτσι ώστε να προσαρμοσω τις κλάσεις αυτές στον κώδικά μου και στις ανάγκες του μοτίβου. \\ \hline
        ΙΧ4 & Προγραμματιστής & Να μπορώ να καθορίσω πεδία που θα προστεθούν στις νέες κλάσεις.  & Έτσι ώστε να προσαρμοσω τις κλάσεις αυτές στον κώδικά μου και στις ανάγκες του μοτίβου. \\ \hline
        ΙΧ5 & Προγραμματιστής & Να μπορώ να καθορίσω μεθόδους που θα προστεθούν στις νέες κλάσεις. & Έτσι ώστε να προσαρμοσω τις μεθόδους αυτές στον κώδικά μου και στις ανάγκες του μοτίβου. \\ \hline
        ΙΧ6 & Προγραμματιστής & Να μπορώ να δημιουργήσω αυτόματα τον κώδικα του μοτίβου με βάση τις όποιες παραμετροποιήσεις έχουν γίνει. & Έτσι ώστε να εισάγω αυτόματα στον κώδικά μου το σκελετό του μοτίβου. \\ \hline
        ΙΧ7 & Προγραμματιστής & Να μπορώ να καθορίσω τα ονόματα των διεπαφών που θα δημιουργηθούν αυτόματα. & Έτσι ώστε να προσαρμοσω τις διεπαφές αυτές στον κώδικά μου και στις ανάγκες του μοτίβου. \\ \hline
        ΙΧ8 & Προγραμματιστής & Να μπορώ να καθορίσω μεθόδους που θα προστεθούν στις νέες διεπαφές. & Έτσι ώστε να προσαρμοσω τις διεπαφές αυτές στον κώδικά μου και στις ανάγκες του μοτίβου. \\ \hline
        ΙΧ9 & Προγραμματιστής & Να μπορώ να ακυρώσω την διαδικασία. & Έτσι ώστε να επιστρέψω σε αυτό που έκανα χωρίς να αλλάξω τον κώδικα μου. \\ \hline
		ΙΧ10 & Προγραμματιστής & Να μπορώ να προσθέσω νέες κλάσεις. & Έτσι ώστε να προσαρμοσω το μοτίβο στον κώδικά μου. \\ \hline
		ΙΧ11 & Προγραμματιστής & Να μπορώ να καθορίσω ποίες διεπαφές θα υλοποιούν οι νέες κλάσεις. & Έτσι ώστε να προσαρμοσω τις κλάσεις αυτές στον κώδικά μου και στις ανάγκες του μοτίβου. \\ \hline
    \end{tabular}
\end{table}
\section{Περιπτώσεις χρήσης}
Οι Περιπτώσεις Χρήσης, αφορούν σύνολα διαδοχικών ενεργειών που προσδιορίζουν τη συμπεριφορά του συστήματος και τις λειτουργικές του απαιτήσεις. Αποτελούν μία πιο λεπτομερειακή προσέγγιση των ιστοριών χρήστη. Κάθε περίπτωση χρήσης πρέπει να διαθέτει τουλάχιστον έναν Actor, δηλαδή κάποιον που παίζει έναν ρόλο και αλληλεπιδρά με το σύστημα με τον τρόπο που ορίζει το περιεχόμενο της περίπτωσης χρήσης.
\begin{table}[H]
	\hspace*{-0.2cm}
    \centering
    \scriptsize
	\begin{tabular}{|p{10cm}|}
		\hline
		\textbf{Περίπτωση χρήσης:} Επιλογή κατηγορίας μοτίβου.\\
		\hline
		\textbf{Αναγνωριστικό:} ΠΧ1\\
		\hline	
		\textbf{Προυποθέσεις:}
		\begin{enumerate}
			\item Ο προγραμματιστής χρειάζεται να έχει επιλέξει το έργο που θα εργαστεί.
		\end{enumerate}\\
		\hline
		\textbf{Ροή γεγονότων:} \\ 
		\begin{enumerate}
			\item Η περίπτωση χρήσης ξεκινά όταν ο προγραμματιστής επιλέξει Import pattern, κάτω από το μενού Design pattern builder στο παράθυρο New του eclipse. 
			\item Το σύστημα εμφανίζει έναν οδηγό.
			\item Ο προγραμματιστής επιλέγει την κατηγορία μοτίβου που επιθυμεί.
			%\item Το σύστημα εμφανίζει τα διαθέσιμα μοτίβα της κατηγορίας αυτής.
		\end{enumerate}\\
		\hline
		\textbf{Μετα-συνθήκες:} \\
		\hline
    \end{tabular}
    \caption{Επιλογή κατηγορίας μοτίβου.}
    \label{tab:selectPatternCategoryUC}
\end{table}
\begin{table}[H]
	\hspace*{-0.2cm}
    \centering
    \scriptsize
	\begin{tabular}{|p{10cm}|}
	\hline
		\textbf{Περίπτωση χρήσης:} Επιλογή μοτίβου \\
	\hline
		\textbf{Αναγνωριστικό:} ΠΧ2 \\
	\hline	
		\textbf{Προυποθέσεις:} \\
		\begin{enumerate}
		 \item Ο προγραμματιστής χρειάζεται να έχει επιλέξει κατηγορία μοτίβου.
		\end{enumerate}\\
	\hline
		\textbf{Ροή γεγονότων:} \\
		\begin{enumerate}
			\item Η περίπτωση χρήσης ξεκινάει όταν ο προγραμματιστής κάνει κλικ στην πτυσσόμενη λίστα.
			\item Το σύστημα εμφανίζει τα διαθέσιμα μοτίβα.
		 	\item Ο προγραμματιστής επιλέγει το μοτίβο που επιθυμεί.
		\end{enumerate}\\
	\hline
		\textbf{Μετα-συνθήκες:} \\
	\hline
    \end{tabular}
    \caption{Επιλογή μοτίβου.}
    \label{tab:selectPatternUC}
\end{table}
\begin{table}[H]
	\hspace*{-0.2cm}
    \centering
    \scriptsize
	\begin{tabular}{|p{10cm}|}
	\hline
		\textbf{Περίπτωση χρήσης:} Καθορισμός μεθόδων κλάσης \\
	\hline
		\textbf{Αναγνωριστικό:} ΠΧ3 \\
	\hline	
		\textbf{Προυποθέσεις:} \\
		\begin{enumerate}
		 \item Ο προγραμματιστής χρειάζεται να είναι στο παράθυρο επεξεργασίας της κλάσης.
		\end{enumerate} \\
	\hline
		\textbf{Ροή γεγονότων:} \\
			\begin{enumerate}
		 \item Η περίπτωση χρήσης ξεκινά όταν ο προγραμματιστής κάνει κλικ στο κουμπί Next του παραθύρου επεξεργασίας πεδίων της κλάσης.
		 \item Το σύστημα εμφανίζει ένα νέο παράθυρο με τις μεθόδους της τρέχουσας κλάσης.
		 \item Για κάθε μέθοδο:\begin{enumerate}
						 		 \item Ο προγραμματιστής επεξεργάζεται το όνομα της μεθόδου.
								 \item Ο προγραμματιστής επεξεργάζεται τον επιστρεφόμενο τύπο της μεθόδου.
		 						  \item Ο προγραμματιστής επεξεργάζεται την ορατότητα της μεθόδου.
			 		 		  \end{enumerate}
 		 \item Ο προγραμματιστής κάνει κλικ στο κουμπί finish.
 		 \item Το σύστημα κλείνει το παράθυρο.
		\end{enumerate} \\
	\hline
		\textbf{Μετα-συνθήκες:} \\
		\begin{enumerate}
			\item Το σύστημα ρυθμίζει το όνομα της κλάσης.
			\item Το σύστημα ρυθμίζει τα πεδία της κλάσης.
			\item Το σύστημα ρυθμίζει τις μεθόδους της κλάσης.
		\end{enumerate} \\
	\hline
    \end{tabular}
    \caption{Καθορισμός μεθόδων κλάσης.}
    \label{tab:setClassMethodsUC}
\end{table}
\begin{table}[H]
	\hspace*{-0.2cm}
    \centering
    \scriptsize
	\begin{tabular}{|p{10cm}|}
	\hline
		\textbf{Περίπτωση χρήσης:} Ονοματοδοσία κλάσης. \\
	\hline
		\textbf{Αναγνωριστικό:} ΠΧ4 \\
	\hline	
		\textbf{Προυποθέσεις:} \\
		\begin{enumerate}
		 \item Ο προγραμματιστής χρειάζεται να έχει επιλέξει μοτίβο.
		\end{enumerate} \\
	\hline
		\textbf{Ροή γεγονότων:} \\
		\begin{enumerate}
		 \item Η περίπτωση χρήσης ξεκινά όταν ο προγραμματιστής κάνει κλικ στο κουμπί Next του παραθύρου επιλογής μοτίβου.
		 \item Το σύστημα εμφανίζει ένα νέο παράθυρο με τις κλάσεις και τις διεπαφές του μοτίβου.
		 \item Ο προγραμματιστής επιλέγει την κλάση που επιθυμεί.
		 \item Ο προγραμματιστής κάνει κλικ στο κουμπί edit class.
 		 \item Το σύστημα εμφανίζει ένα νέο παράθυρο όπου ο χρήστης μπορεί να επεξεργαστεί το όνομα της κλάσης.
 		 \item Ο προγραμματιστής επεξεργάζεται το όνομα της κλάσης.
 		 \item Ο προγραμματιστής κάνει κλικ στο κουμπί Next.
 		 \item Το σύστημα εμφανίζει ένα νέο παράθυρο.
		\end{enumerate} \\
	\hline
		\textbf{Μετα-συνθήκες:} \\
	\hline
    \end{tabular}
    \caption{Ονοματοδοσία κλάσης.}
    \label{tab:nameClassUC}
\end{table}
\begin{table}[H]
	\hspace*{-0.2cm}
    \centering
    \scriptsize
	\begin{tabular}{|p{10cm}|}
	\hline
		\textbf{Περίπτωση χρήσης:} Καθορισμός πεδίων κλάσης \\
	\hline
		\textbf{Αναγνωριστικό:} ΠΧ5 \\
	\hline	
		\textbf{Προυποθέσεις:} \\
		\begin{enumerate}
		 \item Ο προγραμματιστής χρειάζεται να έχει επεξεργαστεί το όνομα της κλάσης
		\end{enumerate} \\
	\hline
		\textbf{Ροή γεγονότων:} \\
		\begin{enumerate}
		 \item Η περίπτωση χρήσης ξεκινά όταν ο προγραμματιστής κάνει κλικ στο κουμπί Next του παραθύρου επεξεργασίας ονόματος της κλάσης.
		 \item Το σύστημα εμφανίζει ένα νέο παράθυρο με τα πεδία της τρέχουσας κλάσης.
		 \item Για κάθε μέθοδο:\begin{enumerate}
						 		 \item Ο προγραμματιστής επεξεργάζεται το όνομα του πεδίου.
								 \item Ο προγραμματιστής επεξεργάζεται τον τύπο του πεδίου.
		 						 \item Ο προγραμματιστής επεξεργάζεται την ορατότητα του πεδίου.
			 		 		  \end{enumerate}
 		 \item Ο προγραμματιστής κάνει κλικ στο κουμπί Next.
 		 \item Το σύστημα εμφανίζει ένα νέο παράθυρο.
		\end{enumerate} \\
	\hline
		\textbf{Μετα-συνθήκες:} \\
	\hline
    \end{tabular}
    \caption{Καθορισμός πεδίων κλάσης.}
    \label{tab:setClassFieldsUC}
\end{table}
\begin{table}[H]
	\hspace*{-0.2cm}
    \centering
    \scriptsize
	\begin{tabular}{|p{10cm}|}
	\hline
		\textbf{Περίπτωση χρήσης:} Ονοματοδοσία διεπαφής  \\
	\hline
		\textbf{Αναγνωριστικό:} ΠΧ6 \\
	\hline	
		\textbf{Προυποθέσεις:} \\
		\begin{enumerate}
		 \item Ο προγραμματιστής χρειάζεται να έχει επιλέξει μοτίβο.
		\end{enumerate} \\
	\hline
		\textbf{Ροή γεγονότων:} \\
		\begin{enumerate}
			\item Η περίπτωση χρήσης ξεκινά όταν ο προγραμματιστής κάνει κλικ στο κουμπί Next του παραθύρου επιλογής μοτίβου.
			\item Το σύστημα εμφανίζει ένα νέο παράθυρο με τις κλάσεις και τις διεπαφές του μοτίβου.
			\item Ο προγραμματιστής επιλέγει την διεπαφή που επιθυμεί.
			\item Ο προγραμματιστής κάνει κλικ στο κουμπί edit interface.
			 \item Το σύστημα εμφανίζει ένα νέο παράθυρο όπου ο χρήστης μπορεί να επεξεργαστεί το όνομα της διεπαφής.
			 \item Ο προγραμματιστής επεξεργάζεται το όνομα της διεπαφής.
			 \item Ο προγραμματιστής κάνει κλικ στο κουμπί Next.
			 \item Το σύστημα εμφανίζει ένα νέο παράθυρο.
		\end{enumerate} \\
	\hline
		\textbf{Μετα-συνθήκες:} \\
	\hline
    \end{tabular}
    \caption{Ονοματοδοσία διεπαφής.}
    \label{tab:nameInterfaceUC}
\end{table}
\begin{table}[H]
	\hspace*{-0.2cm}
    \centering
    \scriptsize
	\begin{tabular}{|p{10cm}|}
	\hline
		\textbf{Περίπτωση χρήσης:} Καθορισμός μεθόδων διεπαφής \\
	\hline
		\textbf{Αναγνωριστικό:} ΠΧ7 \\
	\hline	
		\textbf{Προυποθέσεις:} \\
		\begin{enumerate}
			\item Η περίπτωση χρήσης ξεκινά όταν ο προγραμματιστής κάνει κλικ στο κουμπί Next του παραθύρου επεξεργασίας ονόματος της διεπαφής.
			\item Το σύστημα εμφανίζει ένα νέο παράθυρο με τις μεθόδους της τρέχουσας διεπαφής.
			\item Για κάθε μέθοδο:\begin{enumerate}
									 \item Ο προγραμματιστής επεξεργάζεται το όνομα της μεθόδου.
									\item Ο προγραμματιστής επεξεργάζεται τον επιστρεφόμενο τύπο της μεθόδου.
									  \item Ο προγραμματιστής επεξεργάζεται την ορατότητα της μεθόδου.
								   \end{enumerate}
			 \item Ο προγραμματιστής κάνει κλικ στο κουμπί finish.
			 \item Το σύστημα κλείνει το παράθυρο.
		   \end{enumerate} \\
	   \hline
		   \textbf{Μετα-συνθήκες:} \\
		   \begin{enumerate}
			   \item Το σύστημα ρυθμίζει το όνομα της διεπαφής.
			   \item Το σύστημα ρυθμίζει τις μεθόδους της διεπαφής.
		   \end{enumerate} \\
	\hline
    \end{tabular}
    \caption{Καθορισμός μεθόδων διεπαφής.}
    \label{tab:setInterfaceMethodsUC}
\end{table}
\begin{table}[H]
	\hspace*{-0.2cm}
    \centering
    \scriptsize
	\begin{tabular}{|p{10cm}|}
	\hline
		\textbf{Περίπτωση χρήσης:} Εξαγωγή σκελετού επιλεγμένου μοτίβου. \\
	\hline
		\textbf{Αναγνωριστικό:} ΠΧ8 \\ 
	\hline	
		\textbf{Προυποθέσεις:} \\
		%TODO
		\begin{enumerate}
		 \item Ο προγραμματιστής χρειάζεται να έχει επιλέξει κατηγορία μοτίβου.
		\end{enumerate} \\
	\hline
		\textbf{Ροή γεγονότων:} \\
		\begin{enumerate}
		 \item Η Περίπτωση χρήσης ξεκινά όταν ο προγραμματιστής κάνει κλικ στο κουμπί finish στο παράθυρο επιλογής κλάσης ή διεπαφής.
		 \item Το σύστημα δημιουργεί τα απαραίτητα πηγαία αρχεία java στο επιλεγμένο πακέτο.
		 \item Το σύστημα τερματίζει.
		\end{enumerate} \\
	\hline
		\textbf{Μετα-συνθήκες:} \\
	\hline
    \end{tabular}
    \caption{Εξαγωγή σκελετού επιλεγμένου μοτίβου.}
    \label{tab:createPatternUC}
\end{table}
\begin{table}[H]
	\hspace*{-0.2cm}
    \centering
    \scriptsize
	\begin{tabular}{|p{10cm}|}
	\hline
		\textbf{Περίπτωση χρήσης:} Ακύρωση διαδικασίας \\
	\hline
		\textbf{Αναγνωριστικό:} ΠΧ9 \\
	\hline	
		\textbf{Προυποθέσεις:} \\
		%\begin{enumerate}
		 %\item Ο προγραμματιστής χρειάζεται να έχει επιλέξει κατηγορία μοτίβου.
		%\end{enumerate} \\
	\hline
		\textbf{Ροή γεγονότων:} \\
		\begin{enumerate}
		 \item Ο προγραμματιστής μπορεί ανα πάσα στιγμή να ακυρώσει την διαδικασία.
 		 \item Το σύστημα τερματίζει τα παράθυρα.
		\end{enumerate} \\
	\hline
		\textbf{Μετα-συνθήκες:} \\
	\hline
    \end{tabular}
    \caption{Ακύρωση διαδικασίας.}
    \label{tab:cancelUC}
\end{table}
\begin{table}[H]
	\hspace*{-0.2cm}
    \centering
    \scriptsize
	\begin{tabular}{|p{10cm}|}
	\hline
		\textbf{Περίπτωση χρήσης:} Προσθήκη νέας κλάσης  \\
	\hline
		\textbf{Αναγνωριστικό:} ΠΧ10 \\
	\hline	
		\textbf{Προυποθέσεις:} \\
		\begin{enumerate}
		 \item Ο προγραμματιστής χρειάζεται να έχει επιλέξει μοτίβο.
		 \item Το μοτίβο πρέπει να επιτρέπει την εισαγωγή καινούργιας κλάσης.
		\end{enumerate} \\
	\hline
		\textbf{Ροή γεγονότων:} \\
		\begin{enumerate}
			\item Η περίπτωση χρήσης ξεκινά όταν ο προγραμματιστής κάνει κλικ στο κουμπί Next του παραθύρου επιλογής μοτίβου.
			\item Το σύστημα εμφανίζει ένα νέο παράθυρο με τις κλάσεις και τις διεπαφές του μοτίβου.
			\item Ο προγραμματιστής Κάνει κλικ στο κουμπί Add Class.
			\item Το σύστημα προσθέτει μία νέα κλάση.
			
		\end{enumerate} \\
	\hline
		\textbf{Μετα-συνθήκες:} \\
	\hline
    \end{tabular}
    \caption{Προσθήκη νέας κλάσης.}
    \label{tab:addNewClass}
\end{table}
\begin{table}[H]
	\hspace*{-0.2cm}
    \centering
    \scriptsize
	\begin{tabular}{|p{10cm}|}
	\hline
		\textbf{Περίπτωση χρήσης:} Καθορισμός υλοποιημένης διεπαφής  \\
	\hline
		\textbf{Αναγνωριστικό:} ΠΧ11 \\
	\hline	
		\textbf{Προυποθέσεις:} \\
		\begin{enumerate}
		 \item Ο προγραμματιστής χρειάζεται να έχει επιλέξει μοτίβο.
		\end{enumerate} \\
	\hline
		\textbf{Ροή γεγονότων:} \\
		\begin{enumerate}
			\item Η περίπτωση χρήσης ξεκινά όταν ο προγραμματιστής κάνει κλικ στο κουμπί Next του παραθύρου επιλογής μοτίβου.
			\item Το σύστημα εμφανίζει ένα νέο παράθυρο με τις κλάσεις και τις διεπαφές του μοτίβου.
			\item Ο προγραμματιστής επιλέγει μία κλάση που έχει προσθέσει ο ίδιος.
			\item Ο προγραμματιστής κάνει κλικ στο κουμπί edit class.
			\item Το σύστημα εμφανίζει ένα νέο παράθυρο όπου ο χρήστης μπορεί να επεξεργαστεί την διεπαφή που μπορεί να υλοποιεί η κλάση αυτή.
			\item Ο προγραμματιστής επεξεργάζεται την διεπαφή που μπορεί να υλοποιεί η κλάση αυτή.
			\item Ο προγραμματιστής κάνει κλικ στο κουμπί Next.
			\item Το σύστημα εμφανίζει ένα νέο παράθυρο.
		\end{enumerate} \\
	\hline
		\textbf{Μετα-συνθήκες:} \\
	\hline
    \end{tabular}
    \caption{Καθορισμός υλοποιημένης διεπαφής.}
    \label{tab:setImplementedInterface}
\end{table}
\label{sec:useCases}
\chapter{Σχεδίαση και αρχιτεκτονική λογισμικού}
\label{ch:architecture}
Στην ενότητα αυτή θα περιγραφούν τα βασικά συστατικά του συστήματος
 καθώς και οι σχέσεις που τα συνδέουν μεταξύ τους.
\section{Πακέτα συστήματος}
\label{sec:packages}
\begin{figure}[H]
    \centering
    \includegraphics[width=1.0\textwidth]{Figures/packages.png}
    \label{fig:packageUML}
    \caption{Διάγραμμα UML Πακέτων συστήματος}
\end{figure}
Το λογισμικό αποτελείται από τα παρακάτω πακέτα,
\begin{itemize}
    \item dpb.wizards.mainWizard, σε αυτό το πακέτο, περιέχονται οι κλάσεις που αφορούν το γραφικό περιβάλλον του εργαλείου,
     πιο συγκεκριμένα, οι κλάσεις αυτού του πακέτου έχουν να κάνουν με την επιλογή του μοτίβου και την διαχείριση των κλάσεων και διεπαφών του 
     επιλεγμένου μοτίβου
    \item dpb.wizards.setupWizards, σε αυτό το πακέτο, υπάρχουν οι κλάσεις που έχουν σχέση με την γραφική διεπαφή και την τροποποίηση των κλάσεων και διεπαφών, 
    όπως και των μεθόδων και πεδίων
    \item dpb.controller, αποτελεί το πακέτο μέσω του οποίου η γραφική διεπαφή επικοινωνεί με το back-end. Χρησιμοποιεί τις λειτουργίες που παρέχει το back-end, 
    ώστε να  μετασχηματίζει τα ακατέργαστα δεδομένα των μοτίβων που παίρνει από το υποσύστημα io σε αντικείμενα κλάσεων που παρέχει το πακέτο model
    \item dpb.io, αποτελεί το κομμάτι του συστήματος το οποίο χαρακτηρίζεται ως back-end
     και είναι υπεύθυνο για το διάβασμα του αρχείου xml στο οποίο περιγράφονται τα μοτίβα. Παρέχει λειτουργίες για την εισαγωγή των μοτίβων στο σύστημα
    \item dpb.model, Το πακέτο αυτό, περιέχει τις κλάσεις οι οποίες αναπαριστούν τα αντικείμενα του πεδίου του προβλήματος
    \item dpb.exceptions, Τέλος, στο πακέτο αυτό υπάρχουν κάποιες κλάσεις οι οποίες αναπαριστούν εξαιρέσεις 
    οι οποίες είναι κάποια γεγονότα που συμβαίνουν κατά την εκτέλεση του εργαλείου και διακόπτουν την κανονική ροή του προγράμματος.
\end{itemize}
\newpage
\section{Κλάσεις συστήματος}
\label{sec:classes}
\subsection{Ανάλυση κλάσεων}
\label{subsec:classAnalysis}
Παρακάτω, θα αναλύσουμε κάποιες βασικές κλάσεις του συστήματος,
\begin{itemize}
    \item PatternClass, η κλάση αυτή ανήκει στο πακέτο model και μοντελοποιεί μία κλάση του μοτίβου
    \item PatternInterface, η κλάση αυτή ανήκει στο πακέτο model και μοντελοποιεί μία διεπαφή του μοτίβου
    \item Method, η κλάση αυτή ανήκει στο πακέτο model και μοντελοποιεί μία μέθοδο που ανήκει σε κλάση ή διεπαφή του μοτίβου
    \item Field, η κλάση αυτή ανήκει στο πακέτο model και μοντελοποιεί ένα πεδίο που ανήκει σε κλάση του μοτίβου
    \item FileParser, είναι η βασική κλάση του υποσυστήματος io, είναι υπεύθυνη για το διάβασμα του αρχείου όπου περιγράφονται
    τα μοτίβα και η βασική της δουλειά είναι να διαβάζει το αρχείο xml. Υποστηρίζει λειτουργίες όπως η ανάκτηση 
    των κατηγοριών και των μοτίβων κάθε κατηγορίας, η ανάκτηση των κλάσεων και διεπαφών ενός μοτίβου, 
    η ανάκτηση των annotations κάθε στοιχείου για κάποιο μοτίβο, όπως και οι ιδιότητες ενός μοτίβου 
    οι οποίες είναι εάν επιτρέπεται η εισαγωγή νέας κλάσης, τέλος υποστηρίζει την ανάκτηση μεθόδων και πεδίων.
    \item PatternManager, Μέσω της κλάσης αυτής παρέχεται η δυνατότητα στην γραφική διεπαφή του συστήματος 
    να αντλεί οτιδήποτε χρειάζεται από το υποσύστημα io
    \item PatternGenerator, Περιλαμβάνει την λειτουργία δημιουργίας των πηγαίων αρχείων στο πακέτο του έργου που έχει επιλέξει 
    ο προγραμματιστής ή στο προεπιλεγμένο πακέτο, εάν δεν έχει επιλέξει κάποιο πακέτο, τα οποία περιέχουν 
    τις κλάσεις του μοτίβου με τις μεθόδους και τα πεδία, όπως τα παραμετροποίησε ο προγραμματιστής, καθώς και τις διεπαφές. 
    Επίσης προσθέτει στο classpath του έργου του προγραμματιστή το πακέτο που περιέχει τα annotations 
    ώστε να είναι διαθέσιμα στον προγραμματιστή.
     
\end{itemize}
\subsection{Διάγραμμα κλάσεων}
\begin{figure}[H]
    \centering
    \includegraphics[width=0.9\textwidth]{Figures/system.png}
    \label{fig:systemUML}
    \caption{Διάγραμμα UML βασικών κλάσεων \& διεπαφών συστήματος}
\end{figure}
\begin{figure}[H]
    \centering
    \includegraphics[width=0.7\textwidth]{Figures/mainWizard.png}
    \label{fig:mainWizardUML}
    \caption{Διάγραμμα UML Πακέτου mainWizard}
\end{figure}
\begin{figure}[H]
    \centering
    \includegraphics[width=0.9\textwidth]{Figures/setupWizard.png}
    \label{fig:setupWizardUML}
    \caption{Διάγραμμα UML Πακέτου setupWizards}
\end{figure}
\newpage
\begin{figure}[H]
    \centering
    \includegraphics[width=1.0\textwidth]{Figures/controller.png}
    \label{fig:controllerUML}
    \caption{Διάγραμμα UML Πακέτου controller}
\end{figure}
\begin{figure}[H]
    \centering
    \includegraphics[width=0.4\textwidth]{Figures/io.png}
    \label{fig:ioUML}
    \caption{Διάγραμμα UML Πακέτου io}
\end{figure}
\begin{figure}[H]
    \centering
    \includegraphics[width=1.0\textwidth]{Figures/model.png}
    \label{fig:modelUML}
    \caption{Διάγραμμα UML Πακέτου model}
\end{figure}
\label{subsec:ClassUML}
\section{Κάρτες αρμοδιοτήτων και συνεργασιών κλάσεων}
\label{sec:crc}
\begin{table}[H]
    \centering
    \begin{tabular}{|p{5cm}|p{5cm}|}
        \hline
        \multicolumn{2}{|l|}{Όνομα κλάσης: PatternClass} \\
        \hline
        \textbf{Αρμοδιότητες} & \textbf{Συνεργασίες} \\
        \hline
        \begin{itemize}
            \item Αυτή η κλάση είναι υπεύθυνη για την μοντελοποίηση μίας κλάσης ενός μοτίβου.
        \end{itemize} &   
        \begin{itemize}
            \item Method
            \item Field
        \end{itemize} \\
        \hline
    \end{tabular}
    \label{tab:PatternClassCRC}
    \caption{PatternClass κάρτα αρμοδιοτήτων}
\end{table}
\begin{table}[H]
    \centering
    \begin{tabular}{|p{5cm}|p{5cm}|}
        \hline
        \multicolumn{2}{|l|}{Όνομα κλάσης: PatternInterface} \\
        \hline
        \textbf{Αρμοδιότητες} & \textbf{Συνεργασίες} \\
        \hline
        \begin{itemize}
            \item Αυτή η κλάση είναι υπεύθυνη για την μοντελοποίηση μίας διεπαφής ενός μοτίβου.
        \end{itemize} &   
        \begin{itemize}
            \item Method
        \end{itemize} \\
        \hline
    \end{tabular}
    \label{tab:PatternInterfaceCRC}
    \caption{PatternInterface κάρτα αρμοδιοτήτων}
\end{table}
\begin{table}[H]
    \centering
    \begin{tabular}{|p{5cm}|p{5cm}|}
        \hline
        \multicolumn{2}{|l|}{Όνομα κλάσης: Method} \\
        \hline
        \textbf{Αρμοδιότητες} & \textbf{Συνεργασίες} \\
        \hline
        \begin{itemize}
            \item Η κλάση αυτή διατηρεί τα χαρακτηριστικά μίας μεθόδου που ανήκει σε κάποια κλάση ή διεπαφή.
        \end{itemize} &    
        % \begin{itemize}
        %     \item hi
        % \end{itemize} 
        \\
        \hline
    \end{tabular}
    \label{tab:MethodCRC}
    \caption{Method κάρτα αρμοδιοτήτων}
\end{table}
\begin{table}[H]
    \centering
    \begin{tabular}{|p{5cm}|p{5cm}|}
        \hline
        \multicolumn{2}{|l|}{Όνομα κλάσης: Field} \\
        \hline
        \textbf{Αρμοδιότητες} & \textbf{Συνεργασίες} \\
        \hline
        \begin{itemize}
            \item Η κλάση αυτή διατηρεί τα χαρακτηριστικά ενός πεδίου που ανήκει σε κάποια κλάση.
        \end{itemize} &   
        % \begin{itemize}
        %     \item hi
        % \end{itemize} 
        \\
        \hline
    \end{tabular}
    \label{tab:FieldCRC}
    \caption{Field κάρτα αρμοδιοτήτων}
\end{table}
\begin{table}[H]
    \centering
    \begin{tabular}{|p{10cm}|p{5cm}|}
        \hline
        \multicolumn{2}{|l|}{Όνομα κλάσης: FileParser} \\
        \hline
        \textbf{Αρμοδιότητες} & \textbf{Συνεργασίες} \\
        \hline
        \begin{itemize}
            \item Η κλάση αυτή είναι υπεύθυνη για την άντληση των κατηγοριών που ορίζουν οι GoF \cite{GoF}.
            \item Η κλάση αυτή αρμόδια για την άντληση των μοτίβων κάποιας κατηγορίας.
            \item Η κλάση αυτή είναι αρμόδια για την άντληση των κλάσεων ενός μοτίβου, όπως και την ορατότητα της κλάσης.
            \item Η κλάση αυτή είναι αρμόδια για την άντληση των διεπαφών ενός μοτίβου όπως και την ορατότητα της διεπαφής.
            \item Η κλάση αυτή είναι αρμόδια για την άντληση των μεθόδων κάποιας κλάσης, καθώς τους διάφορους τροποποιητές που μπορεί 
            να έχει μία μέθοδος, όπως και τον επιστρεφόμενο τύπο της μεθόδου και τέλος τον κώδικα της μεθόδου.
            \item Η κλάση αυτή είναι αρμόδια για την άντληση των πεδίων κάποιας κλάσης ενός μοτίβου.
            \item Η κλάση αυτή είναι αρμόδια για την άντληση των μεθόδων κάποιας διεπαφής μαζί με τους 
            τροποποιητές της μεθόδου και τον επιστρεφόμενο τύπο της.
            \item Η κλάση αυτή είναι αρμόδια για την άντληση των annotations ενός στοιχείου του μοτίβου.
        \end{itemize} &   
        % \begin{itemize}
        %     \item 
        % \end{itemize}
         \\
        \hline
    \end{tabular}
    \label{tab:fileParserCRC}
    \caption{FileParser κάρτα αρμοδιοτήτων}
\end{table}
\begin{table}[H]
    \centering
    \begin{tabular}{|p{10cm}|p{5cm}|}
        \hline
        \multicolumn{2}{|l|}{Όνομα κλάσης: PatternManager} \\
        \hline
        \textbf{Αρμοδιότητες} & \textbf{Συνεργασίες} \\
        \hline
        \begin{itemize}
            \item Η κλάση αυτή είναι αρμόδια για την διάθεση των κατηγοριών των μοτίβων στην γραφική διεπαφή.
            \item Η κλάση αυτή είναι αρμόδια για την διάθεση των μοτίβων κάποιας κατηγορίας στην γραφική διεπαφή.
            \item Η κλάση αυτή είναι αρμόδια για την δημιουργία αντικειμένων τύπου PatternClass και την διαθέση τους στο front-end τμήμα του 
            εργαλείου, ώστε να μπορεί να διαχειριστεί ο χρήστης τις διάφορες κλάσεις ενός μοτίβου.
            \item Η κλάση αυτή είναι αρμόδια για την δημιουργία αντικειμένων τύπου PatternInterface και την διαθέση τους στο front-end τμήμα του 
            εργαλείου, ώστε να μπορεί να διαχειριστεί ο χρήστης τις διάφορες διεπαφές ενός μοτίβου.
            \item Η κλάση αυτή είναι αρμόδια για την ενημέρωση του ονόματος μίας κλάσης.
            \item Η κλάση αυτή είναι αρμόδια για την ενημέρωση του ονόματος μίας διεπαφής.
            \item Η κλάση αυτή είναι αρμόδια για την ενημέρωση του ονόματος μίας μεθόδου.
            \item Η κλάση αυτή είναι αρμόδια για την ενημέρωση του ονόματος ενός πεδίου.
        \end{itemize} &   
        \begin{itemize}
            \item FileParser
            \item PatternClass
            \item PatternInterface
            \item Method
            \item Field
            \item Property
        \end{itemize} \\
        \hline
    \end{tabular}
    \label{tab:PatternManagerCRC}
    \caption{PatternManager κάρτα αρμοδιοτήτων}
\end{table}
\begin{table}[H]
    \centering
    \begin{tabular}{|p{5cm}|p{5cm}|}
        \hline
        \multicolumn{2}{|l|}{Όνομα κλάσης: ClassGenerator} \\
        \hline
        \textbf{Αρμοδιότητες} & \textbf{Συνεργασίες} \\
        \hline
        \begin{itemize}
            \item Η κλάση αυτή είναι αρμόδια για την προσθήκη των annotations στο classpath του έργου που έχει επιλέξει ο προγραμματιστής.
            \item Η κλάση αυτή είναι αρμόδια για την παραγωγή των πηγαίων αρχείων java μίας κλάσης.
        \end{itemize} &   
        \begin{itemize}
            \item PatternManager
            \item PatternClass
            \item PatternInterface
            \item Method
            \item Field
            \item Property
        \end{itemize} \\
        \hline
    \end{tabular}
    \label{tab:ClassGeneratorCRC}
    \caption{ClassGenerator κάρτα αρμοδιοτήτων}
\end{table}
\begin{table}[H]
    \centering
    \begin{tabular}{|p{5cm}|p{5cm}|}
        \hline
        \multicolumn{2}{|l|}{Όνομα κλάσης: InterfaceGenerator} \\
        \hline
        \textbf{Αρμοδιότητες} & \textbf{Συνεργασίες} \\
        \hline
        \begin{itemize}
            \item Η κλάση αυτή είναι αρμόδια για την προσθήκη των annotations στο classpath του έργου που έχει επιλέξει ο προγραμματιστής.
            \item Η κλάση αυτή είναι αρμόδια για την παραγωγή των πηγαίων αρχείων java μίας διεπαφής.
        \end{itemize} &   
        \begin{itemize}
            \item PatternManager
            \item PatternClass
            \item PatternInterface
            \item Method
            \item Field
            \item Property
        \end{itemize} \\
        \hline
    \end{tabular}
    \label{tab:InterfaceGeneratorCRC}
    \caption{InterfaceGenerator κάρτα αρμοδιοτήτων}
\end{table}
\chapter{Έλεγχος}
\label{ch:testing}


\section{Έλεγχος δομής σχεδιαστικών μοτίβων}
\label{sec:patternTesting}
\section{Έλεγχος μεθόδων }
\label{sec:patternManagerTesting}
\section{Έλεγχος μεθόδων δημιουργίας πηγαίου κώδικα java}
\label{sec:patternGeneratorTesting}
\chapter{Οδηγός Χρήσης Design Pattern Builder}
\label{ch:manual}


\section{Λειτουργίες χρήστη}
\label{sec:manual}
\begin{figure}[H]
    \centering
    \includegraphics[width=1.0\textwidth]{Figures/open_wizard.png}
    \caption{Άνοιγμα οδηγού.}
    \label{fig:xsd}
\end{figure}
\begin{figure}[H]
    \centering
    \includegraphics[width=1.0\textwidth]{Figures/select_pattern.png}
    \caption{Σελίδα επιλογής κατηγορίας \& μοτίβου.}
    \label{fig:select_pattern}
\end{figure}
\begin{figure}[H]
    \centering
    \includegraphics[width=1.0\textwidth]{Figures/select_pattern1.png}
    \caption{Επιλογή μοτίβου.}
    \label{fig:select_pattern1}
\end{figure}
\begin{figure}[H]
    \centering
    \includegraphics[width=1.0\textwidth]{Figures/classes_interfaces.png}
    \caption{Οι κλάσεις \& διεπαφές του μοτίβου Iterator.}
    \label{fig:classes_interfaces}
\end{figure}
\begin{figure}[H]
    \centering
    \includegraphics[width=1.0\textwidth]{Figures/class_name.png}
    \caption{Επεξεργασία ονόματος κλάσης.}
    \label{fig:class_name}
\end{figure}
\begin{figure}[H]
    \centering
    \includegraphics[width=1.0\textwidth]{Figures/edit_fields.png}
    \caption{Επεξεργασία πεδίων.}
    \label{fig:edit_fields}
\end{figure}
\begin{figure}[H]
    \centering
    \includegraphics[width=1.0\textwidth]{Figures/add_field.png}
    \caption{Προσθήκη πεδίου.}
    \label{fig:add_field}
\end{figure}
\begin{figure}[H]
    \centering
    \includegraphics[width=1.0\textwidth]{Figures/edit_class_methods.png}
    \caption{Επεξεργασία μεθόδων κλάσης.}
    \label{fig:edit_class_methods}
\end{figure}
\begin{figure}[H]
    \centering
    \includegraphics[width=1.0\textwidth]{Figures/add_method.png}
    \caption{Προσθήκη μεθόδου.}
    \label{fig:add_method}
\end{figure}
\begin{figure}[H]
    \centering
    \includegraphics[width=1.0\textwidth]{Figures/edit_parameters.png}
    \caption{Επεξεργασία παραμέτρων μεθόδου.}
    \label{fig:edit_parameters}
\end{figure}
\begin{figure}[H]
    \centering
    \includegraphics[width=1.0\textwidth]{Figures/add_parameters.png}
    \caption{Προσθήκη παραμέτρου.}
    \label{fig:add_parameters}
\end{figure}
\begin{figure}[H]
    \centering
    \includegraphics[width=1.0\textwidth]{Figures/edit_interface.png}
    \caption{Επεξεργασία ονόματος διεπαφής.}
    \label{fig:edit_interface}
\end{figure}
\begin{figure}[H]
    \centering
    \includegraphics[width=1.0\textwidth]{Figures/edit_interface_methods.png}
    \caption{Επεξεργασία μεθόδων διεπαφής.}
    \label{fig:edit_interface_methods}
\end{figure}
\chapter{Επίλογος}
\label{ch:epilogue}
\section{Σύνοψη και συμπεράσματα}
\label{sec:conclusion}
Στη διπλωματική αυτή δημιουργήθηκε ένα καινούριο εργαλείο 
με βασική λειτουργικότητα την ενσωμάτωση σχεδιαστικών μοτίβων σε πηγαίο κώδικα java. 
Το εργαλείο βασίστηκε στα ήδη υπάρχοντα εργαλεία, Pattern Wizard, Patternbox, Design Pattern Automation Toolkit, καθώς και 
Alphaworks Design Pattern Toolkit \cite{PatternBox}, εξαλείφοντας  και συνεισφέροντας  σε βασικές 
τους ελλείψεις όπως η πληθώρα μοτίβων, την παραμετροποίηση των μοτίβων από τον χρήστη και τέλος, τον χαρακτηρισμό 
κάθε παραγόμενης κλάσης και διεπαφής, με την χρήση των annotations.\par
Επιπλέον, κάνοντας μία περιήγηση στο εργαλείο παρατηρούμε τα εξής: όλα τοποθετημένα στο ίδιο παράθυρο με τον κώδικα του χρήστη, 
καθώς πρόκειται για ένα εργαλείο το οποίο είναι επέκταση του περιβάλλοντος Eclipse.
Χάρη στον σχεδιασμό του, μπορεί κάποιος πολύ εύκολα να επεκτείνει τα διαθέσιμα μοτίβα 
και με άλλα μοτίβα εκτός από αυτά των GoF \cite{GoF}. Τέλος, η δυνατότητα των χρηστών να ελέγχουν ανά πάσα ώρα και στιγμή 
την πληροφορία που βρίσκεται μέσα στο εργαλείο, καθώς και οι προσθήκες  νέων μοτίβων από τον υπολογιστή του χρήστη καθώς 
και η διαγραφή τους επιτρέπει την εξατομικευμένη λειτουργία του καλύπτοντας πλήρως τις ανάγκες 
προγραμματιστή ο οποίος έχει καλή γνώση των σχεδιαστικών μοτίβων.
\section{Μελλοντικές επεκτάσεις}
\label{sec:features}
Στην ενότητα αυτή θα προταθούν μερικές επεκτάσεις του εργαλείου. Στο μέλλον θα μπορούσε να προστεθεί η δυνατότητα, 
να μπορεί κάποιος προγραμματιστής να αντιστοιχίσει, κλάσεις και διεπαφές με υπάρχουσες κλάσεις και διεπαφές, 
οι οποίες βρίσκονται στο τρέχον έργο του προγραμματιστή, έτσι ώστε να μπορεί ο προγραμματιστής να προσαρμόσει 
τις ήδη υπάρχουσες κλάσεις και διεπαφές, στις ανάγκες του μοτίβου. Έτσι δίνεται η δυνατότητα, 
σε κάποιον προγραμματιστή να μπορεί εφαρμόσει διάφορα μοτίβα σε έργα τα οποία συντηρεί, 
αναδομώντας τα έργα αυτά και βελτιώνοντας τα.



% Εισαγωγή της βιβλιογραφίας
\addstarredchapterc{\bibname} % minitoc
\bibliographystyle{ieeetr}
\bibliography{Content/Bibliography}

% Προαιρετικά, μπορείτε να εισάγετε παραρτήματα
\appendix
\chapter{Κάρτες αρμοδιοτήτων και συνεργασιών κλάσεων}
\label{app:crc}

\begin{table}[H]
    \centering
    \begin{tabular}{|p{5cm}|p{5cm}|}
        \hline
        \multicolumn{2}{|l|}{Όνομα κλάσης: PatternClass} \\
        \hline
        \textbf{Αρμοδιότητες} & \textbf{Συνεργασίες} \\
        \hline
        \begin{itemize}
            \item Αυτή η κλάση είναι υπεύθυνη για την μοντελοποίηση μίας κλάσης ενός μοτίβου.
        \end{itemize} &   
        \begin{itemize}
            \item Method
            \item Field
        \end{itemize} \\
        \hline
    \end{tabular}
    \caption{PatternClass κάρτα αρμοδιοτήτων.}
    \label{tab:PatternClassCRC}
\end{table}
\begin{table}[H]
    \centering
    \begin{tabular}{|p{5cm}|p{5cm}|}
        \hline
        \multicolumn{2}{|l|}{Όνομα κλάσης: PatternInterface} \\
        \hline
        \textbf{Αρμοδιότητες} & \textbf{Συνεργασίες} \\
        \hline
        \begin{itemize}
            \item Αυτή η κλάση είναι υπεύθυνη για την μοντελοποίηση μίας διεπαφής ενός μοτίβου.
        \end{itemize} &   
        \begin{itemize}
            \item Method
        \end{itemize} \\
        \hline
    \end{tabular}
    \caption{PatternInterface κάρτα αρμοδιοτήτων.}
    \label{tab:PatternInterfaceCRC}
\end{table}
\begin{table}[H]
    \centering
    \begin{tabular}{|p{5cm}|p{5cm}|}
        \hline
        \multicolumn{2}{|l|}{Όνομα κλάσης: Method} \\
        \hline
        \textbf{Αρμοδιότητες} & \textbf{Συνεργασίες} \\
        \hline
        \begin{itemize}
            \item Η κλάση αυτή διατηρεί τα χαρακτηριστικά μίας μεθόδου που ανήκει σε κάποια κλάση ή διεπαφή.
        \end{itemize} &    
        % \begin{itemize}
        %     \item hi
        % \end{itemize} 
        \\
        \hline
    \end{tabular}
    \caption{Method κάρτα αρμοδιοτήτων.}
    \label{tab:MethodCRC}
\end{table}
\begin{table}[H]
    \centering
    \begin{tabular}{|p{5cm}|p{5cm}|}
        \hline
        \multicolumn{2}{|l|}{Όνομα κλάσης: Field} \\
        \hline
        \textbf{Αρμοδιότητες} & \textbf{Συνεργασίες} \\
        \hline
        \begin{itemize}
            \item Η κλάση αυτή διατηρεί τα χαρακτηριστικά ενός πεδίου που ανήκει σε κάποια κλάση.
        \end{itemize} &   
        % \begin{itemize}
        %     \item hi
        % \end{itemize} 
        \\
        \hline
    \end{tabular}
    \caption{Field κάρτα αρμοδιοτήτων.}
    \label{tab:FieldCRC}
\end{table}
\begin{table}[H]
    \centering
    \begin{tabular}{|p{10cm}|p{5cm}|}
        \hline
        \multicolumn{2}{|l|}{Όνομα κλάσης: FileParser} \\
        \hline
        \textbf{Αρμοδιότητες} & \textbf{Συνεργασίες} \\
        \hline
        \begin{itemize}
            \item Η κλάση αυτή είναι υπεύθυνη για την άντληση των κατηγοριών που ορίζουν οι GoF \cite{GoF}.
            \item Η κλάση αυτή αρμόδια για την άντληση των μοτίβων κάποιας κατηγορίας.
            \item Η κλάση αυτή είναι αρμόδια για την άντληση των κλάσεων ενός μοτίβου, όπως και την ορατότητα της κλάσης.
            \item Η κλάση αυτή είναι αρμόδια για την άντληση των διεπαφών ενός μοτίβου όπως και την ορατότητα της διεπαφής.
            \item Η κλάση αυτή είναι αρμόδια για την άντληση των μεθόδων κάποιας κλάσης, καθώς τους διάφορους τροποποιητές που μπορεί 
            να έχει μία μέθοδος, όπως και τον επιστρεφόμενο τύπο της μεθόδου και τέλος τον κώδικα της μεθόδου.
            \item Η κλάση αυτή είναι αρμόδια για την άντληση των πεδίων κάποιας κλάσης ενός μοτίβου.
            \item Η κλάση αυτή είναι αρμόδια για την άντληση των μεθόδων κάποιας διεπαφής μαζί με τους 
            τροποποιητές της μεθόδου και τον επιστρεφόμενο τύπο της.
            \item Η κλάση αυτή είναι αρμόδια για την άντληση των annotations ενός στοιχείου του μοτίβου.
        \end{itemize} &   
        % \begin{itemize}
        %     \item 
        % \end{itemize}
         \\
        \hline
    \end{tabular}
    \caption{FileParser κάρτα αρμοδιοτήτων.}
    \label{tab:fileParserCRC}
\end{table}
\begin{table}[H]
    \centering
    \begin{tabular}{|p{10cm}|p{5cm}|}
        \hline
        \multicolumn{2}{|l|}{Όνομα κλάσης: PatternManager} \\
        \hline
        \textbf{Αρμοδιότητες} & \textbf{Συνεργασίες} \\
        \hline
        \begin{itemize}
            \item Η κλάση αυτή είναι αρμόδια για την διάθεση των κατηγοριών των μοτίβων στην γραφική διεπαφή.
            \item Η κλάση αυτή είναι αρμόδια για την διάθεση των μοτίβων κάποιας κατηγορίας στην γραφική διεπαφή.
            \item Η κλάση αυτή είναι αρμόδια για την δημιουργία αντικειμένων τύπου PatternClass και την διαθέση τους στο front-end τμήμα του 
            εργαλείου, ώστε να μπορεί να διαχειριστεί ο χρήστης τις διάφορες κλάσεις ενός μοτίβου.
            \item Η κλάση αυτή είναι αρμόδια για την δημιουργία αντικειμένων τύπου PatternInterface και την διαθέση τους στο front-end τμήμα του 
            εργαλείου, ώστε να μπορεί να διαχειριστεί ο χρήστης τις διάφορες διεπαφές ενός μοτίβου.
            \item Η κλάση αυτή είναι αρμόδια για την ενημέρωση του ονόματος μίας κλάσης.
            \item Η κλάση αυτή είναι αρμόδια για την ενημέρωση του ονόματος μίας διεπαφής.
            \item Η κλάση αυτή είναι αρμόδια για την ενημέρωση του ονόματος μίας μεθόδου.
            \item Η κλάση αυτή είναι αρμόδια για την ενημέρωση του ονόματος ενός πεδίου.
        \end{itemize} &   
        \begin{itemize}
            \item FileParser
            \item PatternClass
            \item PatternInterface
            \item Method
            \item Field
            \item Property
        \end{itemize} \\
        \hline
    \end{tabular}
    \caption{PatternManager κάρτα αρμοδιοτήτων.}
    \label{tab:PatternManagerCRC}
\end{table}
\begin{table}[H]
    \centering
    \begin{tabular}{|p{5cm}|p{5cm}|}
        \hline
        \multicolumn{2}{|l|}{Όνομα κλάσης: ClassGenerator} \\
        \hline
        \textbf{Αρμοδιότητες} & \textbf{Συνεργασίες} \\
        \hline
        \begin{itemize}
            \item Η κλάση αυτή είναι αρμόδια για την προσθήκη των annotations στο classpath του έργου που έχει επιλέξει ο προγραμματιστής.
            \item Η κλάση αυτή είναι αρμόδια για την παραγωγή των πηγαίων αρχείων java μίας κλάσης.
        \end{itemize} &   
        \begin{itemize}
            \item PatternManager
            \item PatternClass
            \item PatternInterface
            \item Method
            \item Field
            \item Property
        \end{itemize} \\
        \hline
    \end{tabular}
    \label{tab:ClassGeneratorCRC}
    \caption{ClassGenerator κάρτα αρμοδιοτήτων}
\end{table}
\begin{table}[H]
    \centering
    \begin{tabular}{|p{5cm}|p{5cm}|}
        \hline
        \multicolumn{2}{|l|}{Όνομα κλάσης: InterfaceGenerator} \\
        \hline
        \textbf{Αρμοδιότητες} & \textbf{Συνεργασίες} \\
        \hline
        \begin{itemize}
            \item Η κλάση αυτή είναι αρμόδια για την προσθήκη των annotations στο classpath του έργου που έχει επιλέξει ο προγραμματιστής.
            \item Η κλάση αυτή είναι αρμόδια για την παραγωγή των πηγαίων αρχείων java μίας διεπαφής.
        \end{itemize} &   
        \begin{itemize}
            \item PatternManager
            \item PatternClass
            \item PatternInterface
            \item Method
            \item Field
            \item Property
        \end{itemize} \\
        \hline
    \end{tabular}
    \caption{InterfaceGenerator κάρτα αρμοδιοτήτων.}
    \label{tab:InterfaceGeneratorCRC}
\end{table}
%\include{Content/AppendixB}
%\include{Content/AppendixC}


% Εκτύπωση του ευρετηρίου (προαιρετικό)
\printindex


\end{document}
